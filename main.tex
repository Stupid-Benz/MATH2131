%%%%%%%%%%%%%%%%%%%%%%%%%%%%%%%%%%%%%%%%%
% The Legrand Orange Book
% LaTeX Template
% Version 2.1.1 (14/2/16)
%
% This template has been downloaded from:
% http://www.LaTeXTemplates.com
%
% Original author:
% Mathias Legrand (legrand.mathias@gmail.com) with modifications by:
% Vel (vel@latextemplates.com)
%
% License:
% CC BY-NC-SA 3.0 (http://creativecommons.org/licenses/by-nc-sa/3.0/)
%
% Compiling this template:
% This template uses biber for its bibliography and makeindex for its index.
% When you first open the template, compile it from the command line with the 
% commands below to make sure your LaTeX distribution is configured correctly:
%
% 1) pdflatex main
% 2) makeindex main.idx -s StyleInd.ist
% 3) biber main
% 4) pdflatex main x 2
%
% After this, when you wish to update the bibliography/index use the appropriate
% command above and make sure to compile with pdflatex several times 
% afterwards to propagate your changes to the document.
%
% This template also uses a number of packages which may need to be
% updated to the newest versions for the template to compile. It is strongly
% recommended you update your LaTeX distribution if you have any
% compilation errors.
%
% Important note:
% Chapter heading images should have a 2:1 width:height ratio,
% e.g. 920px width and 460px height.
%
%%%%%%%%%%%%%%%%%%%%%%%%%%%%%%%%%%%%%%%%%

%----------------------------------------------------------------------------------------
%	PACKAGES AND OTHER DOCUMENT CONFIGURATIONS
%----------------------------------------------------------------------------------------

\documentclass[11pt,fleqn]{book} % Default font size and left-justified equations

%----------------------------------------------------------------------------------------

\input{structure} % Insert the commands.tex file which contains the majority of the structure behind the template

\begin{document}

%----------------------------------------------------------------------------------------
%	TITLE PAGE
%----------------------------------------------------------------------------------------

\begingroup
\thispagestyle{empty}
\begin{tikzpicture}[remember picture,overlay]
    \coordinate [below=12cm] (midpoint) at (current page.north);
    \node at (current page.north west)
    {\begin{tikzpicture}[remember picture,overlay]
            \node[anchor=north west,inner sep=0pt] at (0,0) {\includegraphics[width=\paperwidth]{background}}; % Background image
            \draw[anchor=north] (midpoint) node [fill=ocre!30!white,fill opacity=0.6,text opacity=1,inner sep=1cm]{\Huge\centering\bfseries\sffamily\parbox[c][][t]{\paperwidth}{\centering The Search for a Title\\[15pt] % Book title
            {\Large A Profound Subtitle}\\[20pt] % Subtitle
            {\huge Dr. John Smith}}}; % Author name
        \end{tikzpicture}};
\end{tikzpicture}
\vfill
\endgroup

%----------------------------------------------------------------------------------------
%	COPYRIGHT PAGE
%----------------------------------------------------------------------------------------

\newpage
~\vfill
\thispagestyle{empty}

\noindent Copyright \copyright\ 2013 John Smith\\ % Copyright notice

\noindent \textsc{Published by Publisher}\\ % Publisher

\noindent \textsc{book-website.com}\\ % URL

\noindent Licensed under the Creative Commons Attribution-NonCommercial 3.0 Unported License (the ``License''). You may not use this file except in compliance with the License. You may obtain a copy of the License at \url{http://creativecommons.org/licenses/by-nc/3.0}. Unless required by applicable law or agreed to in writing, software distributed under the License is distributed on an \textsc{``as is'' basis, without warranties or conditions of any kind}, either express or implied. See the License for the specific language governing permissions and limitations under the License.\\ % License information

\noindent \textit{First printing, March 2013} % Printing/edition date

%----------------------------------------------------------------------------------------
%	TABLE OF CONTENTS
%----------------------------------------------------------------------------------------

%\usechapterimagefalse % If you don't want to include a chapter image, use this to toggle images off - it can be enabled later with \usechapterimagetrue

\chapterimage{chapter_head_1.pdf} % Table of contents heading image

\pagestyle{empty} % No headers

\tableofcontents % Print the table of contents itself

\cleardoublepage % Forces the first chapter to start on an odd page so it's on the right

\pagestyle{fancy} % Print headers again

%----------------------------------------------------------------------------------------
%	PART
%----------------------------------------------------------------------------------------

\part{Part One}

%----------------------------------------------------------------------------------------
%	CHAPTER 1
%----------------------------------------------------------------------------------------

\chapterimage{chapter_head_2.pdf} % Chapter heading image

\chapter{Abstract Vector Spaces}

\section{Binary Operation}\index{Binary Operation}

\begin{definition}[Binary Operation]
    A \emph{binary operation} on a set $S$ is a mapping of the elements of the Cartesian product $S \times S$ to $S$.
    \[ \begin{split}
            f : S \times S & \to S \\ (x,y) &\mapsto f(x,y)
        \end{split}\]
\end{definition}

\begin{example}
    A common example of a binary operation is addition on the set of natural numbers $\mathbb{N}$.
    \begin{equation}
        \begin{split}
            + : \mathbb{N} \times \mathbb{N} & \to \mathbb{N} \\ (x,y) &\mapsto x+y
        \end{split}
    \end{equation}
\end{example}

\begin{definition}[Associative Operation]
    A binary operation $f: S \times S \to S$ is said to be \emph{associative} if, for all $x,y,z \in S$, the following holds:
    \[ f(x,f(y,z)) = f(f(x,y),z) \]
\end{definition}

\begin{example}
    A common example of an associative (binary) operation is addition on the set of natural numbers $\mathbb{N}$. For all $x,y,z \in \mathbb{N}$, we have:
    \begin{equation}
        x + (y + z) = (x + y) + z
    \end{equation}
\end{example}

\newpage

\begin{definition}[Identifiable Operation]
    A binary operation $f: S \times S \to S$ is said to be \emph{identifiable}, or \emph{unital}, if there exists an element $e \in S$, called the \emph{identity element} or \emph{unit element}, such that, for all $x \in S$, the following holds:
    \[ f(x,e) = x = f(e,x) \]
\end{definition}

\begin{example}
    A common example of an identifiable (binary) operation is multiplication on the set of natural numbers $\mathbb{N}$. The identity element is $1$, and for all $x \in \mathbb{N}$, we have:
    \begin{equation}
        x \cdot 1 = x = 1 \cdot x
    \end{equation}
\end{example}

\begin{proposition}
    The identity element of an identifiable operation is unique.
\end{proposition}

\begin{proof}
    Let $e_1$ and $e_2$ be two identity elements for the operation $f$. Then, for any element $x \in S$, we have:
    \[ f(x,e_1) = x = f(e_1,x) \]
    \[ f(x,e_2) = x = f(e_2,x) \]
    Now, consider the element $e_1$:
    \[ f(e_1,e_2) = e_1 \]
    But since $e_2$ is an identity element, we also have:
    \[ f(e_1,e_2) = e_2 \]
    Therefore, we conclude that $e_1 = e_2$, proving the uniqueness of the identity element.
\end{proof}

\begin{remark}
    Two-sided identity must be unique, but one-sided identities need not be.
\end{remark}

\begin{example}
    *** To be asked
\end{example}

\begin{definition}[Inverse Operation]
    A binary operation $f: S \times S \to S$ is said to be \emph{invertible} if, for every element $x \in S$, there exists an element $y \in S$, called the two-sided \emph{inverse} of $x$, denoted as $x^{-1}$, such that:
    \[ f(x,y) = e = f(y,x) \]
    where $e$ is the identity element of the operation.
\end{definition}

\begin{remark}
    Invertible operation exists if inverse operation exists, i.e. there exists an identity element.
\end{remark}

\begin{example}
    A common example of an invertible (binary) operation is addition on the set of integers $\mathbb{Z}$. For every integer $x \in \mathbb{Z}$, there exists an integer $y = -x$ such that:
    \begin{equation}
        x + (-x) = 0 = (-x) + x
    \end{equation}
    where $0$ is the identity element for addition.
\end{example}

\begin{proposition}
    The inverse element of an invertible operation is unique.
\end{proposition}

\begin{proof}
    Let $y_1$ and $y_2$ be two inverses of an element $x \in S$. Then, by definition of inverse, we have:
    \[ f(x,y_1) = e = f(y_1,x) \]
    \[ f(x,y_2) = e = f(y_2,x) \]
    Now, consider the element $y_1$:
    \[ f(y_1,x) = e \]
    But since $y_2$ is also an inverse of $x$, we can substitute $e$ with $f(x,y_2)$:
    \[ f(y_1,x) = f(x,y_2) = e \]
    By the associativity of the operation, we can rearrange this to:
    \[ y_1 = f(y_1, e) = f(y_1,f(x,y_2)) = f(f(y_1,x),y_2) = f(e,y_2) = y_2 \]
    Thus, the inverse element is unique.
\end{proof}

\begin{definition}[Commutative Operation]
    A binary operation $f: S \times S \to S$ is said to be \emph{commutative} if, for all $x,y \in S$, the following holds:
    \[ f(x,y) = f(y,x) \]
\end{definition}

\begin{example}
    A common example of a commutative operation is addition on the set of integers $\mathbb{Z}$. For all $x,y \in \mathbb{Z}$, we have:
    \[ x + y = y + x \]
\end{example}

\begin{definition}[Distributive Operation]
    A binary operation $g: S \times S \to S$ is said to be \emph{distributive} with respect to another binary operation $f: S \times S \to S$ if, for all $x,y,z \in S$, the following holds:
    \[ \begin{split}
        g(x,f(y,z)) &= f(g(x,y),g(x,z)) \\
        g(f(y,z),x) &= f(g(y,x),g(z,x))
    \end{split} \]
\end{definition}

\begin{example}
    A common example of a distributive operation is multiplication over addition on the set of integers $\mathbb{Z}$. For all $x,y,z \in \mathbb{Z}$, we have:
    \[ \begin{split}
        x \cdot (y + z) &= x \cdot y + x \cdot z \\
        (y + z) \cdot x &= y \cdot x + z \cdot x
    \end{split} \]
\end{example}

\newpage

\section{Groups, Rings, Fields}\index{Groups, Rings, Fields}

\begin{definition}[Semigroup]
    A \emph{semigroup} is a set $S$ equipped with an associative binary operation $f: S \times S \to S$.
\end{definition}

\begin{definition}[Monoid]
    A \emph{monoid} is a set $M$ equipped with a binary operation $f: M \times M \to M$ such that the following properties hold:
    \begin{enumerate}
        \item \emph{Closure Property:} For all $x,y \in M$, $f(x,y) \in M$.
        \item \emph{Associative Property}
        \item \emph{Identifiable Property}
    \end{enumerate}
    We say $(M, f)$ is a monoid, and $f$ is the \emph{monoid operation} on the set $M$. A set $M$ with a monoid operation $f$ is the \emph{monoid structure}.
\end{definition}

\begin{definition}[Group]
    A \emph{group} is a set $G$ equipped with a monoid operation $f: G \times G \to G$ with the additional property that every element has an inverse, \emph{Invertible Property}.
\end{definition}

\begin{definition}[Abelian Monoid / Group]
    A monoid / group $(G, f)$ is said to be an \emph{abelian monoid / group} if the monoid / group operation $f$ is commutative, \emph{Commutative Property}.
\end{definition}

\begin{definition}[Ring]
    A ring is a set $R$ equipped with two binary operations $f: R \times R \to R$ (addition) and $g: R \times R \to R$ (multiplication) such that the following properties hold:
    \begin{enumerate}
        \item \emph{Additive Group:} $(R, f)$ is an abelian group.
        \item \emph{Multiplicative Semigroup:} $(R, g)$ is a semigroup.
        \item \emph{Distributive Property:} $g$ with respect to $f$.
    \end{enumerate}
\end{definition}

\begin{definition}[Unital Ring]
    A \emph{unital ring} is a ring $R$ equipped with a multiplicative identity element $1 \in R$ such that for all $x \in R$, $g(1,x) = g(x,1) = x$.
\end{definition}

\begin{definition}[Commutative Ring]
    A \emph{commutative ring} is a ring $R$ such that the multiplication operation $g: R \times R \to R$ is commutative.
\end{definition}

\begin{example}
    $(\mathbb{Z}, +, \times)$ is a unital commutative ring. $(2\mathbb{Z}, +, \times)$ is a commutative ring, but not unital. ($2\mathbb{Z} := {2n \mid n \in \mathbb{Z}}$)
\end{example}

\begin{definition}[Field]
    A \emph{field} is a unital commutative ring $F$ such that every non-zero element has a multiplicative inverse.
\end{definition}

\begin{example}
    $(\mathbb{Q}, +, \times)$, $(\mathbb{R}, +, \times)$ and $(\mathbb{C}, +, \times)$ are fields.
\end{example}

\newpage

%------------------------------------------------

\section{Citation}\index{Citation}

This statement requires citation \cite{book_key}; this one is more specific \cite[122]{article_key}.

%------------------------------------------------

\section{Lists}\index{Lists}

Lists are useful to present information in a concise and/or ordered way\footnote{Footnote example...}.

\subsection{Numbered List}\index{Lists!Numbered List}

\begin{enumerate}
    \item The first item
    \item The second item
    \item The third item
\end{enumerate}

\subsection{Bullet Points}\index{Lists!Bullet Points}

\begin{itemize}
    \item The first item
    \item The second item
    \item The third item
\end{itemize}

\subsection{Descriptions and Definitions}\index{Lists!Descriptions and Definitions}

\begin{description}
    \item[Name] Description
    \item[Word] Definition
    \item[Comment] Elaboration
\end{description}

%----------------------------------------------------------------------------------------
%	CHAPTER 2
%----------------------------------------------------------------------------------------

\chapter{In-text Elements}

\section{Theorems}\index{Theorems}

This is an example of theorems.

\subsection{Several equations}\index{Theorems!Several Equations}
This is a theorem consisting of several equations.

\begin{theorem}[Name of the theorem]
    In $E=\mathbb{R}^n$ all norms are equivalent. It has the properties:
    \begin{align}
         & \big| ||\mathbf{x}|| - ||\mathbf{y}|| \big|\leq || \mathbf{x}- \mathbf{y}||                            \\
         & ||\sum_{i=1}^n\mathbf{x}_i||\leq \sum_{i=1}^n||\mathbf{x}_i||\quad\text{where $n$ is a finite integer}
    \end{align}
\end{theorem}

\subsection{Single Line}\index{Theorems!Single Line}
This is a theorem consisting of just one line.

\begin{theorem}
    A set $\mathcal{D}(G)$ in dense in $L^2(G)$, $|\cdot|_0$.
\end{theorem}

%------------------------------------------------

\section{Definitions}\index{Definitions}

This is an example of a definition. A definition could be mathematical or it could define a concept.

\begin{definition}[Definition name]
    Given a vector space $E$, a norm on $E$ is an application, denoted $||\cdot||$, $E$ in $\mathbb{R}^+=[0,+\infty[$ such that:
    \begin{align}
         & ||\mathbf{x}||=0\ \Rightarrow\ \mathbf{x}=\mathbf{0}        \\
         & ||\lambda \mathbf{x}||=|\lambda|\cdot ||\mathbf{x}||        \\
         & ||\mathbf{x}+\mathbf{y}||\leq ||\mathbf{x}||+||\mathbf{y}||
    \end{align}
\end{definition}

%------------------------------------------------

\section{Notations}\index{Notations}

\begin{notation}
    Given an open subset $G$ of $\mathbb{R}^n$, the set of functions $\varphi$ are:
    \begin{enumerate}
        \item Bounded support $G$;
        \item Infinitely differentiable;
    \end{enumerate}
    a vector space is denoted by $\mathcal{D}(G)$.
\end{notation}

%------------------------------------------------

\section{Remarks}\index{Remarks}

This is an example of a remark.

\begin{remark}
    The concepts presented here are now in conventional employment in mathematics. Vector spaces are taken over the field $\mathbb{K}=\mathbb{R}$, however, established properties are easily extended to $\mathbb{K}=\mathbb{C}$.
\end{remark}

%------------------------------------------------

\section{Corollaries}\index{Corollaries}

This is an example of a corollary.

\begin{corollary}[Corollary name]
    The concepts presented here are now in conventional employment in mathematics. Vector spaces are taken over the field $\mathbb{K}=\mathbb{R}$, however, established properties are easily extended to $\mathbb{K}=\mathbb{C}$.
\end{corollary}

%------------------------------------------------

\section{Propositions}\index{Propositions}

This is an example of propositions.

\subsection{Several equations}\index{Propositions!Several Equations}

\begin{proposition}[Proposition name]
    It has the properties:
    \begin{align}
         & \big| ||\mathbf{x}|| - ||\mathbf{y}|| \big|\leq || \mathbf{x}- \mathbf{y}||                            \\
         & ||\sum_{i=1}^n\mathbf{x}_i||\leq \sum_{i=1}^n||\mathbf{x}_i||\quad\text{where $n$ is a finite integer}
    \end{align}
\end{proposition}

\subsection{Single Line}\index{Propositions!Single Line}

\begin{proposition}
    Let $f,g\in L^2(G)$; if $\forall \varphi\in\mathcal{D}(G)$, $(f,\varphi)_0=(g,\varphi)_0$ then $f = g$.
\end{proposition}

%------------------------------------------------

\section{Examples}\index{Examples}

This is an example of examples.

\subsection{Equation and Text}\index{Examples!Equation and Text}

\begin{example}
    Let $G=\{x\in\mathbb{R}^2:|x|<3\}$ and denoted by: $x^0=(1,1)$; consider the function:
    \begin{equation}
        f(x)=\left\{\begin{aligned}      & \mathrm{e}^{|x|} &  & \text{si $|x-x^0|\leq 1/2$} \\
                     & 0                &  & \text{si $|x-x^0|> 1/2$}\end{aligned}\right.
    \end{equation}
    The function $f$ has bounded support, we can take $A=\{x\in\mathbb{R}^2:|x-x^0|\leq 1/2+\epsilon\}$ for all $\epsilon\in\intoo{0}{5/2-\sqrt{2}}$.
\end{example}

\subsection{Paragraph of Text}\index{Examples!Paragraph of Text}

\begin{example}[Example name]
    \lipsum[2]
\end{example}

%------------------------------------------------

\section{Exercises}\index{Exercises}

This is an example of an exercise.

\begin{exercise}
    This is a good place to ask a question to test learning progress or further cement ideas into students' minds.
\end{exercise}

%------------------------------------------------

\section{Problems}\index{Problems}

\begin{problem}
What is the average airspeed velocity of an unladen swallow?
\end{problem}

%------------------------------------------------

\section{Vocabulary}\index{Vocabulary}

Define a word to improve a students' vocabulary.

\begin{vocabulary}[Word]
    Definition of word.
\end{vocabulary}

%----------------------------------------------------------------------------------------
%	PART
%----------------------------------------------------------------------------------------

\part{Part Two}

%----------------------------------------------------------------------------------------
%	CHAPTER 3
%----------------------------------------------------------------------------------------

\chapterimage{chapter_head_1.pdf} % Chapter heading image

\chapter{Presenting Information}

\section{Table}\index{Table}

\begin{table}[h]
    \centering
    \begin{tabular}{l l l}
        \toprule
        \textbf{Treatments} & \textbf{Response 1} & \textbf{Response 2} \\
        \midrule
        Treatment 1         & 0.0003262           & 0.562               \\
        Treatment 2         & 0.0015681           & 0.910               \\
        Treatment 3         & 0.0009271           & 0.296               \\
        \bottomrule
    \end{tabular}
    \caption{Table caption}
\end{table}

%------------------------------------------------

\section{Figure}\index{Figure}

\begin{figure}[h]
    \centering\includegraphics[scale=0.5]{placeholder}
    \caption{Figure caption}
\end{figure}

%----------------------------------------------------------------------------------------
%	BIBLIOGRAPHY
%----------------------------------------------------------------------------------------

\chapter*{Bibliography}
\addcontentsline{toc}{chapter}{\textcolor{ocre}{Bibliography}}
\section*{Books}
\addcontentsline{toc}{section}{Books}
\printbibliography[heading=bibempty,type=book]
\section*{Articles}
\addcontentsline{toc}{section}{Articles}
\printbibliography[heading=bibempty,type=article]

%----------------------------------------------------------------------------------------
%	INDEX
%----------------------------------------------------------------------------------------

\cleardoublepage
\phantomsection
\setlength{\columnsep}{0.75cm}
\addcontentsline{toc}{chapter}{\textcolor{ocre}{Index}}
\printindex

%----------------------------------------------------------------------------------------

\end{document}
