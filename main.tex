%%%%%%%%%%%%%%%%%%%%%%%%%%%%%%%%%%%%%%%%%
% The Legrand Orange Book
% LaTeX Template
% Version 2.1.1 (14/2/16)
%
% This template has been downloaded from:
% http://www.LaTeXTemplates.com
%
% Original author:
% Mathias Legrand (legrand.mathias@gmail.com) with modifications by:
% Vel (vel@latextemplates.com)
%
% License:
% CC BY-NC-SA 3.0 (http://creativecommons.org/licenses/by-nc-sa/3.0/)
%
% Compiling this template:
% This template uses biber for its bibliography and makeindex for its index.
% When you first open the template, compile it from the command line with the 
% commands below to make sure your LaTeX distribution is configured correctly:
%
% 1) pdflatex main
% 2) makeindex main.idx -s StyleInd.ist
% 3) biber main
% 4) pdflatex main x 2
%
% After this, when you wish to update the bibliography/index use the appropriate
% command above and make sure to compile with pdflatex several times 
% afterwards to propagate your changes to the document.
%
% This template also uses a number of packages which may need to be
% updated to the newest versions for the template to compile. It is strongly
% recommended you update your LaTeX distribution if you have any
% compilation errors.
%
% Important note:
% Chapter heading images should have a 2:1 width:height ratio,
% e.g. 920px width and 460px height.
%
%%%%%%%%%%%%%%%%%%%%%%%%%%%%%%%%%%%%%%%%%

%----------------------------------------------------------------------------------------
%	PACKAGES AND OTHER DOCUMENT CONFIGURATIONS
%----------------------------------------------------------------------------------------

\documentclass[11pt,fleqn]{book} % Default font size and left-justified equations

%----------------------------------------------------------------------------------------

\input{structure} % Insert the commands.tex file which contains the majority of the structure behind the template

\begin{document}

%----------------------------------------------------------------------------------------
%	TITLE PAGE
%----------------------------------------------------------------------------------------

\begingroup
\thispagestyle{empty}
\begin{tikzpicture}[remember picture,overlay]
    \coordinate [below=12cm] (midpoint) at (current page.north);
    \node at (current page.north west)
    {\begin{tikzpicture}[remember picture,overlay]
            \node[anchor=north west,inner sep=0pt] at (0,0) {\includegraphics[width=\paperwidth]{background}}; % Background image
            \draw[anchor=north] (midpoint) node [fill=ocre!30!white,fill opacity=0.6,text opacity=1,inner sep=1cm]{\Huge\centering\bfseries\sffamily\parbox[c][][t]{\paperwidth}{\centering MATH 2131\\[15pt] % Book title
            {\Large Honors in Linear and Abstract Algebra I}\\[20pt] % Subtitle
            {\huge Prof. Meng}}}; % Author name
        \end{tikzpicture}};
\end{tikzpicture}
\vfill
\endgroup

%----------------------------------------------------------------------------------------
%	COPYRIGHT PAGE
%----------------------------------------------------------------------------------------

\newpage
~\vfill
\thispagestyle{empty}

\noindent Copyright \copyright\ 2013 John Smith\\ % Copyright notice

\noindent \textsc{Published by Publisher}\\ % Publisher

\noindent \textsc{book-website.com}\\ % URL

\noindent Licensed under the Creative Commons Attribution-NonCommercial 3.0 Unported License (the ``License''). You may not use this file except in compliance with the License. You may obtain a copy of the License at \url{http://creativecommons.org/licenses/by-nc/3.0}. Unless required by applicable law or agreed to in writing, software distributed under the License is distributed on an \textsc{``as is'' basis, without warranties or conditions of any kind}, either express or implied. See the License for the specific language governing permissions and limitations under the License.\\ % License information

\noindent \textit{First printing, March 2013} % Printing/edition date

%----------------------------------------------------------------------------------------
%	TABLE OF CONTENTS
%----------------------------------------------------------------------------------------

%\usechapterimagefalse % If you don't want to include a chapter image, use this to toggle images off - it can be enabled later with \usechapterimagetrue

\chapterimage{chapter_head_1.pdf} % Table of contents heading image

\pagestyle{empty} % No headers

\tableofcontents % Print the table of contents itself

\cleardoublepage % Forces the first chapter to start on an odd page so it's on the right

\pagestyle{fancy} % Print headers again

%----------------------------------------------------------------------------------------
%	PART
%----------------------------------------------------------------------------------------

\part{Part One}

%----------------------------------------------------------------------------------------
%	CHAPTER 1
%----------------------------------------------------------------------------------------

\chapterimage{chapter_head_2.pdf} % Chapter heading image

\chapter{Abstract Vector Spaces}

\section{Binary Operation}\index{Binary Operation}

\begin{definition}[Binary Operation]
    A \emph{binary operation} on a set $S$ is a mapping of the elements of the Cartesian product $S \times S$ to $S$.
    \[ \begin{split}
            f : S \times S & \to S \\ (x,y) &\mapsto f(x,y)
        \end{split}\]
\end{definition}

\begin{example}
    A common example of a binary operation is addition on the set of natural numbers $\mathbb{N}$.
    \begin{equation}
        \begin{split}
            + : \mathbb{N} \times \mathbb{N} & \to \mathbb{N} \\ (x,y) &\mapsto x+y
        \end{split}
    \end{equation}
\end{example}

\begin{definition}[Associative Operation]
    A binary operation $f: S \times S \to S$ is said to be \emph{associative} if, for all $x,y,z \in S$, 
    \[ f(x,f(y,z)) = f(f(x,y),z) \]
\end{definition}

\begin{example}
    A common example of an associative (binary) operation is addition on the set of natural numbers $\mathbb{N}$. For all $x,y,z \in \mathbb{N}$, we have $x + (y + z) = (x + y) + z$.
\end{example}

\begin{definition}[Identifiable Operation]
    A binary operation $f: S \times S \to S$ is said to be \emph{identifiable}, or \emph{unital}, if there exists an element $e \in S$, the \emph{identity} or \emph{unit element}, such that, for all $x \in S$
    \[ f(x,e) = x = f(e,x) \]
\end{definition}

\begin{example}
    A common example of an identifiable (binary) operation is multiplication on the set of natural numbers $\mathbb{N}$. The identity element is $1$, and for all $x \in \mathbb{N}$, we have $x \cdot 1 = x = 1 \cdot x$.
\end{example}

\begin{proposition}
    The identity element of an identifiable operation is unique.
\end{proposition}

\begin{proof}
    Let $e_1$ and $e_2$ be two identity elements for the operation $f$. Then, for any element $x \in S$, we have:
    \[ f(x,e_1) = x = f(e_1,x) \]
    \[ f(x,e_2) = x = f(e_2,x) \]
    Now, consider the element $e_1$:
    \[ f(e_1,e_2) = e_1 \]
    But since $e_2$ is an identity element, we also have:
    \[ f(e_1,e_2) = e_2 \]
    Therefore, we conclude that $e_1 = e_2$, proving the uniqueness of the identity element.
\end{proof}

\begin{remark}
    Two-sided identity must be unique, but one-sided identities need not be.
\end{remark}

\begin{example}
    
\end{example}

\begin{definition}[Inverse Operation]
    A binary operation $f: S \times S \to S$ is said to be \emph{invertible} if, for every element $x \in S$, there exists an element $y \in S$, called the two-sided \emph{inverse} of $x$, denoted as $x^{-1}$, such that
    \[ f(x,y) = e = f(y,x) \]
    where $e$ is the identity element of the operation.
\end{definition}

\begin{remark}
    Invertible operation exists if inverse operation exists, i.e. there exists an identity element.
\end{remark}

\begin{example}
    A common example of an invertible (binary) operation is addition on the set of integers $\mathbb{Z}$. For every integer $x \in \mathbb{Z}$, there exists an integer $y = -x$ such that:
    \begin{equation}
        x + (-x) = 0 = (-x) + x
    \end{equation}
    where $0$ is the identity element for addition.
\end{example}

\begin{proposition}
    The inverse element of an invertible operation is unique.
\end{proposition}

\begin{proof}
    Let $y_1$ and $y_2$ be two inverses of an element $x \in S$. Then, by definition of inverse, we have:
    \[ f(x,y_1) = e = f(y_1,x) \]
    \[ f(x,y_2) = e = f(y_2,x) \]
    Now, consider the element $y_1$:
    \[ f(y_1,x) = e \]
    But since $y_2$ is also an inverse of $x$, we can substitute $e$ with $f(x,y_2)$:
    \[ f(y_1,x) = f(x,y_2) = e \]
    By the associativity of the operation, we can rearrange this to:
    \[ y_1 = f(y_1, e) = f(y_1,f(x,y_2)) = f(f(y_1,x),y_2) = f(e,y_2) = y_2 \]
    Thus, the inverse element is unique.
\end{proof}

\begin{definition}[Commutative Operation]
    A binary operation $f: S \times S \to S$ is said to be \emph{commutative} if, for all $x,y \in S$, the following holds:
    \[ f(x,y) = f(y,x) \]
\end{definition}

\begin{example}
    A common example of a commutative operation is addition on the set of integers $\mathbb{Z}$. For all $x,y \in \mathbb{Z}$, we have:
    \[ x + y = y + x \]
\end{example}

\begin{definition}[Distributive Operation (Harmonic Property)]
    A binary operation $g: S \times S \to S$ is said to be \emph{distributive} with respect to another binary operation $f: S \times S \to S$ if, for all $x,y,z \in S$, the following holds:
    \[ \begin{split}
        g(x,f(y,z)) &= f(g(x,y),g(x,z)) \\
        g(f(y,z),x) &= f(g(y,x),g(z,x))
    \end{split} \]
\end{definition}

\begin{example}
    A common example of a distributive operation is multiplication over addition on the set of integers $\mathbb{Z}$. For all $x,y,z \in \mathbb{Z}$, we have:
    \[ \begin{split}
        x \cdot (y + z) &= x \cdot y + x \cdot z \\
        (y + z) \cdot x &= y \cdot x + z \cdot x
    \end{split} \]
\end{example}

\newpage

\section{Groups, Rings, Fields}\index{Groups, Rings, Fields}

\begin{definition}[Monoid]
    A \emph{monoid} is a set $M$ equipped with a binary operation $f: M \times M \to M$ such that the following properties hold:
    \begin{enumerate}
        \item \emph{Closure Property:} For all $x,y \in M$, $f(x,y) \in M$.
        \item \emph{Associative Property}
        \item \emph{Identifiable Property}
    \end{enumerate}
    We say $(M, f)$ is a monoid, and $f$ is the \emph{monoid operation} on the set $M$. A set $M$ with a monoid operation $f$ is the \emph{monoid structure}.
\end{definition}

\begin{definition}[Group]
    A \emph{group} is a set $G$ equipped with a monoid operation $f: G \times G \to G$ with the additional property that every element has an inverse, \emph{Invertible Property}.
\end{definition}

\begin{definition}[Abelian Monoid / Group]
    A monoid / group $(G, f)$ is said to be an \emph{abelian monoid / group} if the monoid / group operation $f$ is commutative, \emph{Commutative Property}.
\end{definition}

\begin{definition}[Unital Ring]
    A (unital) ring is a set $R$ equipped with two binary operations $f: R \times R \to R$ (addition) and $g: R \times R \to R$ (multiplication) such that the following properties hold:
    \begin{enumerate}
        \item \emph{Additive Group:} $(R, f)$ is an abelian group.
        \item \emph{Multiplicative Monoid:} $(R, g)$ is a monoid.
        \item \emph{Distributive Property:} $g$ with respect to $f$.
    \end{enumerate}
\end{definition}

\begin{definition}[Commutative Ring]
    A \emph{commutative ring} is a unital ring $R$ such that the multiplication operation $g: R \times R \to R$ is commutative.
\end{definition}

\begin{example}
    $(\mathbb{Z}, +, \times)$ is a unital commutative ring.
\end{example}

\begin{definition}[Field]
    A \emph{field} is a unital commutative ring $F$ such that every non-zero element has a multiplicative inverse.
\end{definition}

\begin{example}
    $(\mathbb{Q}, +, \times)$, $(\mathbb{R}, +, \times)$ and $(\mathbb{C}, +, \times)$ are fields.
\end{example}

\begin{example}
    $(\mathbb{Z}/2\mathbb{Z}, +, \times)$ is a field, where $\mathbb{Z}/2\mathbb{Z} = \{\bar{0}, \bar{1}\}$, $\bar{0}$ is the set of even integers and $\bar{1}$ is the set of odd integers.
    It follows the additions and multiplications below:
    \begin{equation}
        \begin{array}{c|cc}
              + & \bar{0} & \bar{1} \\ \hline
            \bar{0} & \bar{0} & \bar{1} \\
            \bar{1} & \bar{1} & \bar{0}
        \end{array}
        \quad
        \begin{array}{c|cc}
              \times & \bar{0} & \bar{1} \\ \hline
            \bar{0} & \bar{0} & \bar{0} \\
            \bar{1} & \bar{0} & \bar{1}
        \end{array}
    \end{equation}
\end{example}

\newpage

\section{Morphisms}\index{Morphisms}

\begin{definition}[Morphisms]
    A \emph{morphism} is a structure-preserving map between two algebraic structures (e.g., groups, rings, fields). Formally, let $(A, \cdot_A)$ and $(B, \cdot_B)$ be two algebraic structures. A morphism $f: A \to B$ is a function such that:
    \[
        f(x \cdot_A y) = f(x) \cdot_B f(y) \quad \forall x, y \in A
    \]
\end{definition}

\begin{definition}[Monoid Homomorphism]
    A \emph{monoid homomorphism} is a morphism between two monoids that preserves the monoid structure. Formally, let $(M_1, \cdot_1)$ and $(M_2, \cdot_2)$ be two monoids with identity elements $e_1$ and $e_2$, respectively. A function $f: M_1 \to M_2$ is a monoid homomorphism if:
    \begin{enumerate}
        \item $f(x \cdot_1 y) = f(x) \cdot_2 f(y) \quad \forall x, y \in M_1$
        \item $f(e_1) = e_2$
    \end{enumerate}
\end{definition}

\begin{definition}[Group Homomorphism]
    A \emph{group homomorphism} is a morphism between two groups that preserves the group structure. Formally, let $(G_1, \cdot_1)$ and $(G_2, \cdot_2)$ be two groups with identity elements $e_1$ and $e_2$, respectively. A function $f: G_1 \to G_2$ is a group homomorphism if:
    \begin{enumerate}
        \item $f(x \cdot_1 y) = f(x) \cdot_2 f(y) \quad \forall x, y \in G_1$
        \item $f(e_1) = e_2$
        \item $f(x^{-1}) = (f(x))^{-1} \quad \forall x \in G_1$
    \end{enumerate}
\end{definition}

\begin{proposition}
    The second and third properties of a group homomorphism are consequences of the first property.
\end{proposition}

\begin{proof}
    Let $f: G_1 \to G_2$ be a group homomorphism satisfying the first property. We will show that the second and third properties follow from it.

    \textbf{Second Property:} To show that $f(e_1) = e_2$, we use the fact that $e_1$ is the identity element in $G_1$. For any element $x \in G_1$, we have:
    \[
        f(x) = f(x \cdot_1 e_1) = f(x) \cdot_2 f(e_1)
    \]
    Since $f(x)$ is an arbitrary element in $G_2$, this implies that $f(e_1)$ must be the identity element in $G_2$, i.e., $f(e_1) = e_2$.

    \textbf{Third Property:} To show that $f(x^{-1}) = (f(x))^{-1}$ for all $x \in G_1$, we use the fact that $x^{-1}$ is the inverse of $x$ in $G_1$. We have:
    \[
        e_2 = f(e_1) = f(x \cdot_1 x^{-1}) = f(x) \cdot_2 f(x^{-1})
    \]
    This shows that $f(x^{-1})$ is the inverse of $f(x)$ in $G_2$, i.e., $f(x^{-1}) = (f(x))^{-1}$.

    Therefore, both the second and third properties of a group homomorphism are indeed consequences of the first property.
\end{proof}

\begin{remark}
    For monoid homomorphisms, the second property cannot be derived from the first property. Consider the identity element $e_1$ in $M_1$. If we apply the first property, we get $f(e_1 \cdot_1 e_1) = f(e_1) \cdot_2 f(e_1)$. This simplifies to $f(e_1) = f(e_1) \cdot_2 f(e_1)$, which does not necessarily imply that $f(e_1)$ is the identity element in $M_2$, i.e., $f(e_1) \neq e_2$, but $f(e_1)$ is the idempotent element in $M_2$. Therefore, the second property must be explicitly stated for monoid homomorphisms.

    However in the case of group homomorphisms, the existence of inverses ensures that there is only one element that can be idempotent under the group operation, which is the identity element. Thus, for group homomorphisms, the second property can be derived from the first property.
\end{remark}

\begin{definition}[Ring Homomorphism]
    A \emph{ring homomorphism} is a morphism between two rings that preserves both the additive and multiplicative structures. Formally, let $(R_1, +_1, \cdot_1)$ and $(R_2, +_2, \cdot_2)$ be two rings with identity elements $0_1$, $1_1$ and $0_2$, $1_2$, respectively. A function $f: R_1 \to R_2$ is a ring homomorphism if:
    \begin{enumerate}
        \item $f(x +_1 y) = f(x) +_2 f(y) \quad \forall x, y \in R_1$
        \item $f(x \cdot_1 y) = f(x) \cdot_2 f(y) \quad \forall x, y \in R_1$
        \item $f(1_1) = 1_2$
    \end{enumerate}    
\end{definition}

\begin{definition}[Endomorphism]
    An \emph{endomorphism} is a morphism from an algebraic structure to itself. Formally, let $(A, \cdot)$ be an algebraic structure. An endomorphism $f: A \to A$ is a function such that:
    \[
        f(x \cdot y) = f(x) \cdot f(y) \quad \forall x, y \in A
    \]
\end{definition}

\begin{definition}[Endomorphism Ring]
    The set of all endomorphisms of an abelian group $(G, +)$, denoted by $E((G, +))$, forms a (non-commutative) ring under pointwise addition and composition of functions. The addition operation is defined as:
    \[
        (f + g)(x) = f(x) + g(x) \quad \forall x \in G
    \]
    The multiplication operation is defined as:
    \[
        (f \circ g)(x) = f(g(x)) \quad \forall x \in G
    \]
    The identity element for addition is the zero endomorphism, which maps every element to the identity element of the group. ($0: A \to A \quad x \mapsto 0$) The identity element for multiplication is the identity endomorphism, which maps every element to itself. ($1: A \to A \quad x \mapsto x$)
    
\end{definition}

\newpage

\section{Vector Spaces}\index{Vector Spaces}

\begin{definition}[Linear Structure over a Field]
    A \emph{linear structure} over a field $F$ is a pair $((V, +), \cdot)$ where $(V, +)$ is an abelian group with a ring homomorphism $F \to E((V, +))$, where $E((V, +))$ is the endomorphism ring of the abelian group $(V, +)$.
    \[ \begin{split}
            \cdot : F &\to E((V, +)) \\
            \alpha &\mapsto \alpha\cdot : (x \mapsto \alpha \cdot x)
        \end{split}
    \]
    The ring homomorphism is an action of the field $F$ on the abelian group $(V, +)$, called \emph{scalar multiplication}.
\end{definition}


\newpage

%------------------------------------------------

\section{Citation}\index{Citation}

This statement requires citation \cite{book_key}; this one is more specific \cite[122]{article_key}.

%------------------------------------------------

\section{Lists}\index{Lists}

Lists are useful to present information in a concise and/or ordered way\footnote{Footnote example...}.

\subsection{Numbered List}\index{Lists!Numbered List}

\begin{enumerate}
    \item The first item
    \item The second item
    \item The third item
\end{enumerate}

\subsection{Bullet Points}\index{Lists!Bullet Points}

\begin{itemize}
    \item The first item
    \item The second item
    \item The third item
\end{itemize}

\subsection{Descriptions and Definitions}\index{Lists!Descriptions and Definitions}

\begin{description}
    \item[Name] Description
    \item[Word] Definition
    \item[Comment] Elaboration
\end{description}

%----------------------------------------------------------------------------------------
%	CHAPTER 2
%----------------------------------------------------------------------------------------

\chapter{In-text Elements}

\section{Theorems}\index{Theorems}

This is an example of theorems.

\subsection{Several equations}\index{Theorems!Several Equations}
This is a theorem consisting of several equations.

\begin{theorem}[Name of the theorem]
    In $E=\mathbb{R}^n$ all norms are equivalent. It has the properties:
    \begin{align}
         & \big| ||\mathbf{x}|| - ||\mathbf{y}|| \big|\leq || \mathbf{x}- \mathbf{y}||                            \\
         & ||\sum_{i=1}^n\mathbf{x}_i||\leq \sum_{i=1}^n||\mathbf{x}_i||\quad\text{where $n$ is a finite integer}
    \end{align}
\end{theorem}

\subsection{Single Line}\index{Theorems!Single Line}
This is a theorem consisting of just one line.

\begin{theorem}
    A set $\mathcal{D}(G)$ in dense in $L^2(G)$, $|\cdot|_0$.
\end{theorem}

%------------------------------------------------

\section{Definitions}\index{Definitions}

This is an example of a definition. A definition could be mathematical or it could define a concept.

\begin{definition}[Definition name]
    Given a vector space $E$, a norm on $E$ is an application, denoted $||\cdot||$, $E$ in $\mathbb{R}^+=[0,+\infty[$ such that:
    \begin{align}
         & ||\mathbf{x}||=0\ \Rightarrow\ \mathbf{x}=\mathbf{0}        \\
         & ||\lambda \mathbf{x}||=|\lambda|\cdot ||\mathbf{x}||        \\
         & ||\mathbf{x}+\mathbf{y}||\leq ||\mathbf{x}||+||\mathbf{y}||
    \end{align}
\end{definition}

%------------------------------------------------

\section{Notations}\index{Notations}

\begin{notation}
    Given an open subset $G$ of $\mathbb{R}^n$, the set of functions $\varphi$ are:
    \begin{enumerate}
        \item Bounded support $G$;
        \item Infinitely differentiable;
    \end{enumerate}
    a vector space is denoted by $\mathcal{D}(G)$.
\end{notation}

%------------------------------------------------

\section{Remarks}\index{Remarks}

This is an example of a remark.

\begin{remark}
    The concepts presented here are now in conventional employment in mathematics. Vector spaces are taken over the field $\mathbb{K}=\mathbb{R}$, however, established properties are easily extended to $\mathbb{K}=\mathbb{C}$.
\end{remark}

%------------------------------------------------

\section{Corollaries}\index{Corollaries}

This is an example of a corollary.

\begin{corollary}[Corollary name]
    The concepts presented here are now in conventional employment in mathematics. Vector spaces are taken over the field $\mathbb{K}=\mathbb{R}$, however, established properties are easily extended to $\mathbb{K}=\mathbb{C}$.
\end{corollary}

%------------------------------------------------

\section{Propositions}\index{Propositions}

This is an example of propositions.

\subsection{Several equations}\index{Propositions!Several Equations}

\begin{proposition}[Proposition name]
    It has the properties:
    \begin{align}
         & \big| ||\mathbf{x}|| - ||\mathbf{y}|| \big|\leq || \mathbf{x}- \mathbf{y}||                            \\
         & ||\sum_{i=1}^n\mathbf{x}_i||\leq \sum_{i=1}^n||\mathbf{x}_i||\quad\text{where $n$ is a finite integer}
    \end{align}
\end{proposition}

\subsection{Single Line}\index{Propositions!Single Line}

\begin{proposition}
    Let $f,g\in L^2(G)$; if $\forall \varphi\in\mathcal{D}(G)$, $(f,\varphi)_0=(g,\varphi)_0$ then $f = g$.
\end{proposition}

%------------------------------------------------

\section{Examples}\index{Examples}

This is an example of examples.

\subsection{Equation and Text}\index{Examples!Equation and Text}

\begin{example}
    Let $G=\{x\in\mathbb{R}^2:|x|<3\}$ and denoted by: $x^0=(1,1)$; consider the function:
    \begin{equation}
        f(x)=\left\{\begin{aligned}      & \mathrm{e}^{|x|} &  & \text{si $|x-x^0|\leq 1/2$} \\
                     & 0                &  & \text{si $|x-x^0|> 1/2$}\end{aligned}\right.
    \end{equation}
    The function $f$ has bounded support, we can take $A=\{x\in\mathbb{R}^2:|x-x^0|\leq 1/2+\epsilon\}$ for all $\epsilon\in\intoo{0}{5/2-\sqrt{2}}$.
\end{example}

\subsection{Paragraph of Text}\index{Examples!Paragraph of Text}

\begin{example}[Example name]
    \lipsum[2]
\end{example}

%------------------------------------------------

\section{Exercises}\index{Exercises}

This is an example of an exercise.

\begin{exercise}
    This is a good place to ask a question to test learning progress or further cement ideas into students' minds.
\end{exercise}

%------------------------------------------------

\section{Problems}\index{Problems}

\begin{problem}
What is the average airspeed velocity of an unladen swallow?
\end{problem}

%------------------------------------------------

\section{Vocabulary}\index{Vocabulary}

Define a word to improve a students' vocabulary.

\begin{vocabulary}[Word]
    Definition of word.
\end{vocabulary}

%----------------------------------------------------------------------------------------
%	PART
%----------------------------------------------------------------------------------------

\part{Part Two}

%----------------------------------------------------------------------------------------
%	CHAPTER 3
%----------------------------------------------------------------------------------------

\chapterimage{chapter_head_1.pdf} % Chapter heading image

\chapter{Presenting Information}

\section{Table}\index{Table}

\begin{table}[h]
    \centering
    \begin{tabular}{l l l}
        \toprule
        \textbf{Treatments} & \textbf{Response 1} & \textbf{Response 2} \\
        \midrule
        Treatment 1         & 0.0003262           & 0.562               \\
        Treatment 2         & 0.0015681           & 0.910               \\
        Treatment 3         & 0.0009271           & 0.296               \\
        \bottomrule
    \end{tabular}
    \caption{Table caption}
\end{table}

%------------------------------------------------

\section{Figure}\index{Figure}

\begin{figure}[h]
    \centering\includegraphics[scale=0.5]{placeholder}
    \caption{Figure caption}
\end{figure}

%----------------------------------------------------------------------------------------
%	BIBLIOGRAPHY
%----------------------------------------------------------------------------------------

\chapter*{Bibliography}
\addcontentsline{toc}{chapter}{\textcolor{ocre}{Bibliography}}
\section*{Books}
\addcontentsline{toc}{section}{Books}
\printbibliography[heading=bibempty,type=book]
\section*{Articles}
\addcontentsline{toc}{section}{Articles}
\printbibliography[heading=bibempty,type=article]

%----------------------------------------------------------------------------------------
%	INDEX
%----------------------------------------------------------------------------------------

\cleardoublepage
\phantomsection
\setlength{\columnsep}{0.75cm}
\addcontentsline{toc}{chapter}{\textcolor{ocre}{Index}}
\printindex

%----------------------------------------------------------------------------------------

\end{document}
