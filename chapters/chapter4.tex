\chapter{Multilinear Algebras}

\section{Tensor Products}

We have learnt what is a bilinear map between linear spaces. However, bilinear maps are not linear maps, so we cannot directly apply the tools we have developed for linear maps to bilinear maps. To fix this problem, we introduce the notion of \emph{tensor products}, which ``linearises'' bilinear maps. There are several ways to characterise tensor products.

\subsection{Characterisation via Universal Property}
We start with the following universal property that characterises tensor products.
\begin{proposition}[Universal Property of Tensor Product]\label{prop:universal_property_tensor_product}
  Let $V$ and $W$ be linear spaces over $F$. For any linear space $Z$ and any bilinear map $\phi \colon V \times W \to Z$, there exists a unique linear map $\widetilde{\phi} \colon V \otimes W \to Z$ such that the following diagram commutes:
  \begin{center}
    \begin{tikzcd}
      V \times W \arrow[r, hook, "\tau"] \arrow[rd, "\phi"] & V \otimes W \arrow[d, dashed, "\widetilde{\phi}"] \\
      & Z
    \end{tikzcd}
  \end{center}
\end{proposition}
\begin{proof}
  If such a map $\widetilde{\phi}$ exists, then for any $v \in V$ and $w \in W$, we must have
  \[
    \widetilde{\phi}(\tau(v, w)) = \widetilde{\phi}(v \otimes w) = \phi(v, w).
  \]
  Since elements of the form $v \otimes w$ span $V \otimes W$, this determines $\widetilde{\phi}$ uniquely. To show existence, we define $\widetilde{\phi}$ on the basis elements by $\widetilde{\phi}(v \otimes w) := \phi(v, w)$ for all $v \in V, w \in W$, and extend linearly to all of $V \otimes W$. It is straightforward to verify that $\widetilde{\phi}$ is linear and makes the diagram commute.
\end{proof}

This proposition shows that any bilinear map from $V \times W$ to another linear space $Z$ can be uniquely factored through the tensor product $V \otimes W$, i.e., $\Bil(V \times W, Z) \simeq \Hom(V \otimes W, Z)$. Moreover, we have the natural isomorphism between $\Bil(V \times W, -)$ and $\Hom(V, \Hom(W, -))$ in Example~\ref{ex:bilinear_natural_isomorphism}. Thus, we have the following natural isomorphism:
\[
  \Hom(V \otimes W, -) \simeq \Hom(V, \Hom(W, -)).
\]
This shows that the tensor product functor $- \otimes W$ is left adjoint to the Hom functor $\Hom(W, -)$, i.e., we have the following adjoint pair of endofunctors on $\Vect_F$:
\begin{center}
  \begin{tikzcd}[labels=auto]
    \Vect_F \arrow[r, "- \otimes W", yshift=0.5ex] & \Vect_F \arrow[l, "{\Hom(W, -)}", yshift=-0.5ex]
  \end{tikzcd}
\end{center}
Such an adjunction is called a \emph{tensor-Hom adjunction}.

\subsection{Categorical Characterisation}
There is actually another way to characterise tensor products using categories. Consider the category $\Bilin(V, W)$ whose objects are bilinear maps from $V \times W$ to linear spaces in $\Vect_F$, and whose morphisms are commutative diagrams of the following form:
\begin{center}
  \begin{tikzcd}[column sep=normal]
    & V \times W \arrow[ld, "\phi"] \arrow[rd, "\psi"] & \\
    Z \arrow[rr, "f"] & & Z'
  \end{tikzcd}
\end{center}
where $\phi \colon V \times W \to Z$ and $\psi \colon V \times W \to Z'$ are bilinear maps, and $f \colon Z \to Z'$ is a linear map such that $f \circ \phi = \psi$. Then the tensor product $V \otimes W$ together with the bilinear map $\tau \colon V \times W \to V \otimes W$ is a \emph{initial object} in the category $\Bilin(V, W)$, i.e., for any bilinear map $\phi \colon V \times W \to Z$, there exists a unique morphism from $\tau$ to $\phi$ in $\Bilin(V, W)$.

\subsection{Construction of Tensor Products via Quotient Spaces}
Recall that we can construct free vector spaces using sets. We can use this idea to construct tensor products. Consider the free vector space $F[V \times W]$ generated by the set $V \times W$. Then elements of $F[V \times W]$ are finite linear combinations of elements of the form $(v, w)$ for $v \in V, w \in W$. However, $F[V \times W]$ is too large to be the tensor product $V \otimes W$, as it does not satisfy the bilinearity conditions. To fix this, we take the quotient of $F[V \times W]$ by the subspace $R$ spanned by all elements of the following forms:
\begin{align*}
   & (v_1 + v_2, w) - (v_1, w) - (v_2, w), \\
   & (v, w_1 + w_2) - (v, w_1) - (v, w_2), \\
   & (\alpha v, w) - \alpha (v, w),        \\
   & (v, \alpha w) - \alpha (v, w),
\end{align*}
for all $v, v_1, v_2 \in V$, $w, w_1, w_2 \in W$, and $\alpha \in F$. The reason for taking this quotient is to enforce the bilinearity conditions. Then we define the tensor product $V \otimes W$ as the quotient space:
\[
  V \otimes W := F[V \times W] / R.
\]
The bilinear map $\tau \colon V \times W \to V \otimes W$ is defined by $\tau(v, w) = [(v, w)]$, the equivalence class of $(v, w)$ in the quotient space. It is straightforward to verify that this construction satisfies the universal property of tensor products. We can conclude this with the following diagram:
\begin{center}
  \begin{tikzcd}[column sep=normal]
    & F[V \times W] \arrow[rd, ustblue, two heads, "\pi"] \\[-1.8em]
    V \times W \arrow[ur, ustblue, hook, "\iota"] \arrow[rr, ustred, hook, "\tau"] \arrow[rrd, ustred, "\phi"] &  & V \otimes W \arrow[d, ustred, dashed, "\widetilde{\phi}"] \\
    & & Z
  \end{tikzcd}
\end{center}
Here, the blue arrows represent the construction of the tensor product, where $\iota$ is the inclusion map and $\pi$ is the quotient map. The red arrows represent the universal property of the tensor product, where $\phi$ is any bilinear map from $V \times W$ to $Z$, and $\widetilde{\phi}$ is the unique linear map from $V \otimes W$ to $Z$ that makes the diagram commute. The dashed blue arrow $\overline{\phi}$ represents the composition of $\phi$ with $\iota$, which factors through the quotient map $\pi$ to give $\widetilde{\phi}$.

Moreover, the inclusion map $\tau$ is `surjective' in the sense that the images of elements of the form $\tau(v, w) = v \otimes w$ span the entire tensor product $V \otimes W$.

\subsection{Tensor Products of $k$ Linear Spaces}
The tensor product can be generalised to more than two linear spaces. Given $k$ linear spaces $V_1, V_2, \ldots, V_k$ over a field $F$, their tensor product $V_1 \otimes V_2 \otimes \cdots \otimes V_k$ is defined similarly using the universal property. Moreover, the tensor product is associative and commutative up to natural isomorphisms, i.e.,
\[
  V_1 \otimes V_2 \otimes V_3 \simeq (V_1 \otimes V_2) \otimes V_3 \simeq V_1 \otimes (V_2 \otimes V_3), \quad V_1 \otimes V_2 \simeq V_2 \otimes V_1.
\]
We can consider the following diagram to illustrate the associativity and commutativity of tensor products:
\begin{center}
  \begin{tikzcd}
    V_1 \times V_2 \times V_3 \arrow[r, hook, "\tau_1"] \arrow[d, hook, "\tau_2"] & (V_1 \otimes V_2) \times V_3 \arrow[d, hook, "\tau_3"] \\
    V_1 \times (V_2 \otimes V_3) \arrow[r, hook, "\tau_4"] & V_1 \otimes V_2 \otimes V_3
  \end{tikzcd}
  \qquad
  \begin{tikzcd}
    V_1 \times V_2 \arrow[r, hook, "\tau"] \arrow[d, "\sigma"] & V_1 \otimes V_2 \arrow[d, <->, dashed, "\widetilde{\sigma}"] \\
    V_2 \times V_1 \arrow[r, hook, "\tau'"] & V_2 \otimes V_1
  \end{tikzcd}
\end{center}
Here $\sigma$ is the swap map that sends $(v_1, v_2)$ to $(v_2, v_1)$. The $\widetilde{\sigma}$ is the induced isomorphism between $V_1 \otimes V_2$ and $V_2 \otimes V_1$ that makes the diagram commute.

\subsection{Properties of Tensor Products}
We have the following natural isomorphisms for tensor products of linear spaces:
\[
  - \otimes F \simeq \id_{\Vect_F} \simeq F \otimes - \simeq \Hom(F, -) \simeq {(-)}^{**}.
\]

We have the following natural isomorphism for tensor products of finite-dimensional linear spaces:
\[
  \Hom(V, W \otimes Z) \simeq \Hom(V, W) \otimes Z.
\]
This shows that the Hom functor $\Hom(V, -)$ commutes with tensor products. However, in general, the tensor product functor $- \otimes Z$ does not commute with the Hom functor $\Hom(V, -)$. One reason is that the tensor product functor is a left adjoint, while the Hom functor is a right adjoint; left adjoints do not generally commute with right adjoints. Another reason is that the tensor product is defined via quotienting by certain \emph{finite} linear combinations, while the Hom functor may involve \emph{infinite} linear combinations when the domain is infinite-dimensional.

This natural isomorphism can be used to prove $\Hom(V, W) \simeq V^* \otimes W$ for finite-dimensional linear spaces $V$ and $W$. We can see this by setting $Z = F$ in the above natural isomorphism:
\[
  \Hom(V, W \otimes F) \simeq \Hom(V, W) \otimes F.
\]
Since $W \otimes F \simeq W$ and $\Hom(V, W) \otimes F \simeq \Hom(V, W)$, we have
\[
  \Hom(V, W) \simeq V^* \otimes W.
\]

Another one is ${(V \otimes W)}^* \simeq V^* \otimes W^*$ for finite-dimensional linear spaces $V$ and $W$. We can see this by setting $Z = F$ in the tensor-Hom adjunction:
\[
  \Hom(V \otimes W, F) \simeq \Hom(V, \Hom(W, F)).
\]
Since $\Hom(W, F) \simeq W^*$ and $\Hom(V, W^*) \simeq V^* \otimes W^*$, we have
\[
  {(V \otimes W)}^* \simeq V^* \otimes W^*.
\]

By considering $V \otimes W \simeq \Hom(V^*, W)$, we can deduce the dimension formula for tensor products:
\[
  \dim(V \otimes W) = \dim(\Hom(V^*, W)) = \dim(V^*) \cdot \dim(W) = \dim(V) \cdot \dim(W).
\]

Under the natural isomorphism $\End(V) \simeq {(\End(V))}^*$, we can identify the trace map $\tr \colon \End(V) \to F$ with the identity map $\id_V$ in $\End(V)$. This gives us a categorical interpretation of the trace map as the evaluation of the identity endomorphism.

We also have the distribution of tensor products and Hom-set over direct sums:
\begin{align*}
  V \otimes (W_1 \oplus W_2) & \simeq (V \otimes W_1) \oplus (V \otimes W_2). \\
  \Hom(V, W_1 \oplus W_2)    & \simeq \Hom(V, W_1) \oplus \Hom(V, W_2),       \\
  \Hom(V_1 \oplus V_2, W)    & \simeq \Hom(V_1, W) \oplus \Hom(V_2, W).
\end{align*}
In categorical terms, consider the category $\Vect_F$ as the category of finite-dimensional linear spaces over $F$ and $\Vect_F^{\op}$ as its opposite category. Then the first isomorphism shows that there is a natural isomorphism between functors from $\Vect_F \times \Vect_F \times \Vect_F$ to $\Vect_F$. The second isomorphism shows that there is a natural isomorphism between functors from $\Vect_F^{\op} \times \Vect_F \times \Vect_F$ to $\Vect_F$. The third isomorphism shows that there is a natural isomorphism between functors from $\Vect_F^{\op} \times \Vect_F^{\op} \times \Vect_F$ to $\Vect_F$.

\section{Tensors}
In this section, we introduce tensors as multilinear maps and multilinear forms, and explore the change of basis for tensors. We also define tensor spaces and discuss some special types of tensors, such as symmetric and skew-symmetric tensors.

\begin{definition}[Multilinear Map]\label{def:multilinear_map}
  A \emph{multilinear map} between finite-dimensional $F$-linear spaces $V_1, V_2, \ldots, V_k$ and $W$ is a map $\phi \colon V_1 \times V_2 \times \cdots \times V_k \to W$ ($k$ times) that is linear in each argument when the other arguments are held fixed; that is, for each $1 \leq i \leq k$, and for all $v_j \in V_j$ ($j \neq i$), the map
  \[
    V_i \to W, \quad v_i \mapsto \phi(v_1, v_2, \ldots, v_k)
  \]
  is a \hyperref[def:linear_map]{linear map}.
\end{definition}

\begin{definition}[Multilinear Form]\label{def:multilinear_form}
  A \emph{multilinear form}, or \emph{covariant tensor} or \emph{$k$-tensor} or \emph{$k$-linear form}, on a finite-dimensional $F$-linear space $V$ is a \hyperref[def:multilinear_map]{multilinear map} $\phi \colon V^k \to F$ ($k$ times). It is an element of the $k$-th \hyperref[def:tensor_power]{tensor power} of the dual space $V^*$, i.e., $\phi \in {(V^*)}^{\otimes k}$.
\end{definition}

Specifically, a linear form is simply a linear functional on $V$, i.e., an element of the dual space $V^*$. A bilinear form is an element of $V^* \otimes V^*$. To show that the set of all bilinear forms on $V$ is isomorphic to $V^* \otimes V^*$, we consider the following diagram:
\begin{center}
  \begin{tikzcd}
    \Bil(V \times V, F) \arrow[r, <->] & \Hom(V, V^*) \arrow[r, <->] & V^* \otimes V^*
  \end{tikzcd}
\end{center}
Here, the first isomorphism is given by the currying process, and the second isomorphism is given by the natural isomorphism $\Hom(V, W) \simeq V^* \otimes W$ for finite-dimensional $F$-linear spaces $V$ and $W$.

Moreover, we have the following two important special cases of 2-linear forms.
\begin{definition}[Symmetric Bilinear Form]\label{def:symmetric_bilinear_form}
  A \emph{symmetric bilinear form} on a finite-dimensional $F$-linear space $V$ is a \hyperref[def:multilinear_form]{multilinear form} $\phi \colon V \times V \to F$ such that $\phi(v, w) = \phi(w, v)$ for all $v, w \in V$. It is an element of the \hyperref[def:symmetric_power]{second symmetric power} of the dual space $V^*$, i.e., $\phi \in \Sy^2(V^*)$.
\end{definition}
\begin{definition}[Skew-symmetric Bilinear Form]\label{def:skew_symmetric_bilinear_form}
  A \emph{skew-symmetric bilinear form} on a finite-dimensional $F$-linear space $V$ is a \hyperref[def:multilinear_form]{multilinear form} $\phi \colon V \times V \to F$ such that $\phi(v, w) = -\phi(w, v)$ for all $v, w \in V$. It is an element of the \hyperref[def:exterior_power]{second exterior power} of the dual space $V^*$, i.e., $\phi \in \Lambda^2(V^*)$.
\end{definition}
\begin{remark}
  Here, we assume that the characteristic of the field $F$ is not 2, so that the relation $\phi(v, w) = -\phi(w, v)$ is equivalent to $\phi(v, v) = 0$ for all $v, w \in V$. If the characteristic of $F$ is 2, then the skew-symmetric bilinear form would be defined by the relation $\phi(v, v) = 0$ for all $v \in V$.
\end{remark}

Then we can define tensor spaces.
\begin{definition}[Tensor Space]\label{def:tensor_space}
  The \emph{tensor space} of type $(k, l)$ on a finite-dimensional $F$-linear space $V$, denoted by $T^{k, l}(V)$, is defined as the tensor product of the $k$-th \hyperref[def:tensor_power]{tensor power} of $V$ and the $l$-th \hyperref[def:tensor_power]{tensor power} of the dual space $V^*$:
  \[
    T^{k, l}(V) = V^{\otimes k} \otimes {(V^*)}^{\otimes l}.
  \]
  Elements of $T^{k, l}(V)$ are called \emph{tensors of type $(k, l)$} on $V$, where $k$ is the number of \emph{contravariant} indices and $l$ is the number of \emph{covariant} indices.
\end{definition}

If a tensor is of type $(k, 0)$, then it is called a \emph{contravariant tensor} or simply a \emph{$k$-tensor}, and it is an element of the $k$-th \hyperref[def:tensor_power]{tensor power} of $V$, i.e., $\Ten^{k, 0}(V) = V^{\otimes k}$. If a tensor is of type $(0, l)$, then it is called a \emph{covariant tensor} or simply an \emph{$l$-form}, and it is an element of the $l$-th \hyperref[def:tensor_power]{tensor power} of the dual space $V^*$, i.e., $\Ten^{0, l}(V) = {(V^*)}^{\otimes l}$.

Given that $\End(V) \simeq V \otimes V^* = \Ten^{1, 1}(V)$, we can see that tensors of type $(1, 1)$ can be interpreted as linear operators on $V$ and represented by $a^i_j$ in a basis. Here, the contravariant index $i$ represents the row index, and the covariant index $j$ represents the column index. To get the matrix representation of the linear operator with respect to a basis, we have
\[
  a^i_j = \langle \hat{e}^i, A(\vec{e}_j) \rangle,
\]
where $\{\vec{e}_j\}$ is a basis of $F^n$ and $\{\hat{e}^i\}$ is the dual basis of $F^n$. Then we can identify the linear operator $T$ with tensor of type $(1, 1)$ as follows:
\[
  T \simeq T^i_j v_i \otimes v^j.
\]
Here, the $\{ v_i \}$ form a basis of $V$, and the $\{ v^j \}$ form the dual basis of $V^*$.

An object is considered as \emph{covariant} if it transforms like the basis vectors under a change of basis, and it is considered as \emph{contravariant} if it transforms oppositely to the basis vectors under a change of basis. We have the following transformation tables for covariant and contravariant objects under a change of basis.
\begin{tabularx}{\textwidth}{X X}
  \toprule
  \bfseries Object                    & \bfseries Transformation Type \\
  \midrule
  Standard Basis Vector ($\vec{e}_i$) & Covariant                     \\
  \midrule
  Dual Basis Vector ($\hat{e}^i$)     & Contravariant                 \\
  \midrule
  Component of a Vector ($v^i$)       & Contravariant                 \\
  \midrule
  Component of a Covector ($v_i$)     & Covariant                     \\
  \bottomrule
\end{tabularx}

In general, an element $t \in \Ten^{r, s}(V)$ can be represented in a basis as follows:
\[
  t = t^{i_1 i_2 \cdots i_r}_{j_1 j_2 \cdots j_s} v_{i_1} \otimes v_{i_2} \otimes \cdots \otimes v_{i_r} \otimes v^{j_1} \otimes v^{j_2} \otimes \cdots \otimes v^{j_s},
\]
where the $\{ v_i \}$ form a basis of $V$, and the $\{ v^j \}$ form the dual basis of $V^*$. The representation depends on the choice of basis, i.e., the following two representations are equivalent under a change of basis:
\[
  {\left[t^{i_1 i_2 \cdots i_r}_{j_1 j_2 \cdots j_s}\right]}_{\B_V} \simeq {\left[\widetilde{t}^{\widetilde{i}_1 \widetilde{i}_2 \cdots \widetilde{i}_r}_{\widetilde{j}_1 \widetilde{j}_2 \cdots \widetilde{j}_s}\right]}_{\widetilde{\B_V}}
\]
The two representations are related by the change of basis matrices as follows:
\[
  (\widetilde{v}_1, \widetilde{v}_2, \cdots, \widetilde{v}_n) = (v_1, v_2, \cdots, v_n) A,
\]
where $A$ is the change of basis matrix, $A = {[a^i_{\widetilde{j}}]}_{\B_V}^{\widetilde{\B_V}}$, which is in the general linear group of $V$, $\GL(V)$. It is an automorphism group of $V$ that consists of all invertible linear maps from $V$ to itself. We will learn more later.
\begin{remark}
  It is the right action of $\GL(V)$ on the set of bases of $V$:
  \[
    \B_V \times \GL(V) \to \B_V, \quad (\B_V, A) \mapsto \B_V A.
  \]
\end{remark}
Then we have the following transformation rule for the components of the tensor under the change of basis:
\[
  \widetilde{v}_{\widetilde{j}} = v_i a^i_{\widetilde{j}}, \quad v_k = \widetilde{v}_{\widetilde{j}} b^{\widetilde{j}}_k,
\]
where $B$ is the inverse of $A$, i.e., $B = A^{-1}$, and $B = {[b^{\widetilde{i}}_j]}_{\widetilde{\B_V}}^{\B_V}$. We have $a^i_{\widetilde{j}} b_j^{\widetilde{k}} = \delta^{\widetilde{k}}_{\widetilde{j}}$ and $b_j^{\widetilde{i}} a^j_{\widetilde{k}} = \delta^{\widetilde{i}}_{\widetilde{k}}$.
\begin{remark}
  For easier memorisation, our professor suggests using the following partial derivative notation to represent the change of basis matrices:
  \[
    \frac{\partial \widetilde{v}_{\widetilde{j}}}{\partial v_i} = a^i_{\widetilde{j}}, \quad \frac{\partial v_k}{\partial \widetilde{v}_{\widetilde{j}}} = b_k^{\widetilde{j}}
  \]
  To memorise it, we consider the lower indices in denominators (lower) will flip to the upper indices in numerators. (As lower twice, so flip to upper)

  Then we can use the chain rule to verify the two equations of $A$ and $A^{-1}$:
  \[
    \frac{\partial \widetilde{v}_{\widetilde{j}}}{\partial v_i} \frac{\partial v_k}{\partial \widetilde{v}_{\widetilde{j}}} = \delta^i_k
  \]
\end{remark}

Then we have the transformation rule for the representation of $t \in \mathcal{T}^{r, s} V$ under the base change from $\B_V$ to $\widetilde{\B_V}$:
\[
  \widetilde{t}^{{\color{ustblue} \widetilde{i}_1 \widetilde{i}_2 \cdots \widetilde{i}_r}}_{{\color{red} \widetilde{j}_1 \widetilde{j}_2 \cdots \widetilde{j}_s}} = \left({\color{ustblue} b_{i_1}^{\widetilde{i}_1} b_{i_2}^{\widetilde{i}_2} \cdots b_{i_r}^{\widetilde{i}_r}}\right) t^{{\color{ustblue} i_1 i_2 \cdots i_r}}_{{\color{ustred} j_1 j_2 \cdots j_s}} \left({\color{ustred} a^{j_1}_{\widetilde{j}_1} a^{j_2}_{\widetilde{j}_2} \cdots a^{j_s}_{\widetilde{j}_s}}\right)
\]
Given that $\B_V = \{ \vec{v}_1, \cdots, \vec{v}_n \}$ is a basis of $V$, then we can define a basis of $\mathcal{T}^{r, s} V$ as follows:
\[
  \B_{\mathcal{T}^{r, s} V} = \{ \vec{v}_{i_1} \otimes \vec{v}_{i_2} \otimes \cdots \otimes \vec{v}_{i_r} \otimes \hat{v}^{j_1} \otimes \hat{v}^{j_2} \otimes \cdots \otimes \hat{v}^{j_s} : 1 \leq i_1, i_2, \cdots, i_r, j_1, j_2, \cdots, j_s \leq n \}
\]
For symmetric and skew-symmetric tensors, we have:
\begin{align*}
  \B_{\Sy^k V}     & = \{ v_{i_1} v_{i_2} \cdots v_{i_k} \mid 1 \leq i_1 \leq i_2 \leq \cdots \leq i_k \leq n \},             \\
  \B_{\Lambda^k V} & = \{ v_{i_1} \wedge v_{i_2} \wedge \cdots \wedge v_{i_k} \mid 1 \leq i_1 < i_2 < \cdots < i_k \leq n \}.
\end{align*}
However, there is a more elegant way to define the basis of symmetric and skew-symmetric tensors. We just need to ensure that the representation of any symmetric tensor is unique for a given basis. For example, for symmetric bilinear forms, we can add the condition that the components remain the same under the swap of any two indices:
\[
  t = t^{ij} v_i v_j = t^{ji} v_j v_i \implies t^{ij} = t^{ji}.
\]
Thus, we can define the basis of symmetric bilinear forms as follows:
\[
  \B_{\Sy^2 V} = \{ v_i v_j \mid 1 \leq i, j \leq n \}.
\]
Similarly, for skew-symmetric tensors, we can add the condition that the components change sign under the swap of any two indices:
\[
  t = t^{ij} v_i \wedge v_j = -t^{ji} v_j \wedge v_i \implies t^{ij} = -t^{ji}.
\]
Thus, we can define the basis of skew-symmetric bilinear forms as follows:
\[
  \B_{\Lambda^2 V} = \{ v_i \wedge v_j \mid 1 \leq i < j \leq n \}.
\]

In conclusion, we have to make sure that the representation of any symmetric or skew-symmetric multilinear form is unique for a given basis by the following conditions respectively:
\begin{align*}
   & \text{Symmetric:} \quad t^{i_1 i_2 \cdots i_k} = t^{i_{\sigma(1)} i_{\sigma(2)} \cdots i_{\sigma(k)}}                   \\
   & \text{Skew-symmetric:} \quad t^{i_1 i_2 \cdots i_k} = \sgn(\sigma) t^{i_{\sigma(1)} i_{\sigma(2)} \cdots i_{\sigma(k)}}
\end{align*}
The $\sigma$ is any permutation in the symmetric group $S_k$. We will learn more about symmetric groups later.

\section{Multilinear Algebras}

\subsection{Algebras}
Before introducing tensor algebras, we first define algebras over fields.

\begin{definition}[Algebraic Structure]\label{def:algebraic_structure}
  An \emph{algebraic structure} on a $F$-linear space $A$ is a \hyperref[def:bilinear_map]{bilinear map} $\mu \colon A \times A \to A$, called the \emph{multiplication} on $A$; or equivalently, a \hyperref[def:linear_map]{linear map} $\mu \colon A \otimes A \to A$. The pair $(A, \mu)$ is called a $F$-\emph{algebra}.
\end{definition}

\begin{example}
  The set of all polynomials with coefficients in $F$, denoted by $F[x]$, forms an algebra over $F$ with the usual polynomial multiplication. The multiplication map $\mu \colon F[x] \times F[x] \to F[x]$ is defined by $\mu(p(x), q(x)) = p(x) \cdot q(x)$ for all $p(x), q(x) \in F[x]$. Moreover, $F[x]$ is a unital commutative associative algebra over $F$, where the multiplicative identity is the constant polynomial $1$.
\end{example}

\begin{example}
  The set of all square matrices of size $n$ with entries in $F$, denoted by $\Mat_n(F)$, forms an algebra over $F$ with the usual matrix multiplication. The multiplication map $\mu \colon M_n(F) \times M_n(F) \to M_n(F)$ is defined by $\mu(A, B) = AB$ for all $A, B \in M_n(F)$. Moreover, $M_n(F)$ is a unital non-commutative associative algebra over $F$, where the multiplicative identity is the identity matrix $I_n$. However, $M_n(F)$ is generally not commutative for $n \geq 2$.
\end{example}

\begin{example}
  The 3-dimensional Euclidean space $\R^3$ forms an algebra over $\R$ with the cross product $\times \colon \R^3 \times \R^3 \to \R^3$ as multiplication. The multiplication map $\mu \colon \R^3 \times \R^3 \to \R^3$ is defined by $\mu(\mathbf{u}, \mathbf{v}) = \mathbf{u} \times \mathbf{v}$ for all $\mathbf{u}, \mathbf{v} \in \R^3$. Moreover, $\R^3$ is a non-unital non-commutative non-associative algebra over $\R$.
\end{example}

\begin{remark}
  $(\R^3, \times)$ is an example of a simple real Lie algebra. It is the Lie algebra of the Lie group $\SO(3)$, the special orthogonal group in dimension 3, i.e., the 3-dimensional rotations. $(\R^3, \times)$ is denoted by $\mathfrak{so}(3)$. It is the Lie algebra of the infinitesimal symmetries of a pointed 3-dimensional Euclidean space.
\end{remark}

\begin{definition}[Lie Algebra]\label{def:lie_algebra}
  A \emph{Lie algebra} over $F$ is an algebra $\mathfrak{g}$ over $F$ with the \emph{Lie bracket} $[\cdot, \cdot] \colon \mathfrak{g} \times \mathfrak{g} \to \mathfrak{g}$ as the multiplication satisfying the following properties for all $x, y, z \in \mathfrak{g}$:
  \begin{enumerate}
    \item (Skew-symmetry) $[x, x] = 0$;
    \item (Jacobi Identity) $[x, [y, z]] + [y, [z, x]] + [z, [x, y]] = 0$.
  \end{enumerate}
  The skew-symmetry property implies that $[x, y] = -[y, x]$ for all $x, y \in \mathfrak{g}$ if the \hyperref[def:characteristic_of_a_field]{characteristic} of $F$ is not 2.
\end{definition}

\begin{definition}[Algebra Homomorphism]\label{def:algebra_homomorphism}
  An \emph{algebra homomorphism} between two algebras $(A, \mu)$ and $(B, \nu)$ over the same field $F$ is a \hyperref[def:linear_map]{linear map} $\phi \colon A \to B$ that respects the algebraic structure; that is, for all $u, v \in A$, the following property holds:
  \[
    \phi(\mu(u, v)) = \nu(\phi(u), \phi(v)).
  \]
\end{definition}

\begin{definition}[Graded Linear Space]\label{def:graded_linear_space}
  A $\Z_{\geq 0}$-\emph{graded linear space} over $F$ is a linear space $V$ over $F$ together with a decomposition of $V$ into a \hyperref[def:internal_direct_sum]{direct sum} of \hyperref[def:linear_subspace]{linear subspaces} indexed by the integers:
  \[
    V_{\smallbullet} = \bigoplus_{n \in \Z_{\geq 0}} V_n,
  \]
  where each $V_n$ is called the \emph{degree $n$ component} of $V$. An element $v \in V_n$ is said to be \emph{homogeneous of degree $n$}.
\end{definition}

\begin{definition}[Graded Linear Map]\label{def:graded_linear_map}
  A \emph{graded linear map} with graded degree $k$ $\phi \colon V_{\smallbullet} \to W_{\smallbullet}$ is a \hyperref[def:linear_map]{linear map} if $\phi(V_n) \subseteq W_{n+k}$ for all $n \in \Z_{\geq 0}$.
\end{definition}

\subsection{Tensor Algebras}
We are now ready to define tensor algebras.

\begin{definition}[Tensor Power]\label{def:tensor_power}
  The \emph{$k$-th tensor power} of a finite-dimensional $F$-linear space $V$, denoted by $V^{\otimes k}$, is defined as follows:
  \[
    V^{\otimes k} = \underbrace{V \otimes V \otimes \cdots \otimes V}_{k \text{ times}}
  \]
  for all $k \geq 0$. By convention, we define $V^{\otimes 0} = F$.
\end{definition}

The dimension of the $k$-th tensor power is given by the following formula:
\[ \dim(V^{\otimes k}) = {(\dim(V))}^k \]

\begin{definition}[Tensor Algebra]\label{def:tensor_algebra}
  The \emph{tensor algebra} of a finite-dimensional $F$-linear space $V$ is the $\Z_{\geq 0}$-\hyperref[def:graded_linear_space]{graded linear space} over $F$ defined as the \hyperref[def:external_direct_sum]{external direct sum} of all \hyperref[def:tensor_power]{tensor powers} of $V$:
  \[
    \T(V) = \bigoplus_{k = 0}^{\infty} V^{\otimes k} = F \oplus V \oplus (V \otimes V) \oplus (V \otimes V \otimes V) \oplus \cdots
  \]
  equipped with the natural multiplication defined by the tensor product, making $\T(V)$ an algebra over $F$. The multiplication map is defined as follows:
  \begin{align*}
    \otimes \colon \T(V) \times \T(V)     & \to \T(V)                             \\
    \left( \sum_n u_n, \sum_m v_m \right) & \mapsto \sum_{n, m} (u_n \otimes v_m)
  \end{align*}
  where $u_n \in V^{\otimes n}$ and $v_m \in V^{\otimes m}$ for all $n, m \geq 0$.
\end{definition}

\begin{remark}
  As the algebra product is bilinear, it suffices to know the product of two homogeneous elements, i.e., $V^{\otimes n} \times V^{\otimes m} \to \T(V)$ for all $n, m \geq 0$. So $\T(V)$ is a $\Z_{\geq 0}$-graded algebra over $F$. As the tensor algebra is bi-additive, we have the following equality:
  \[
    \sum_n u_n \otimes \sum_m v_m = \sum_n (u_n \otimes \sum_m v_m) = \sum_n \sum_m (u_n \otimes v_m) = \sum_{n, m} (u_n \otimes v_m)
  \]
\end{remark}

To show that the multiplication map $\otimes \colon \T(V) \times \T(V) \to \T(V)$ is well-defined, we consider the following diagram:
\begin{center}
  \begin{tikzcd}[row sep=normal, column sep=normal]
    V^{\otimes n} \times V^{\otimes m} \arrow[rr] \arrow[dd] & & \T(V) \\
    & V^{\otimes (n + m)} \arrow[ur, hook] & \\
    V^{\otimes n} \otimes V^{\otimes m} \arrow[ur, dashed, "\tau"]
  \end{tikzcd}
\end{center}
Then, we have to show the existence of $\tau$. Again, we can consider the following diagram:
\begin{center}
  \begin{tikzcd}
    V^n \times V^m \arrow[r, ustblue, hook] & V^{\otimes n} \times V^{\otimes m} \arrow[d, ustblue, hook] \\
    V^{n + m} \arrow[u, ustblue, <->] \arrow[r, ustred, hook] \arrow[dr, ustred, hook] & V^{\otimes n} \otimes V^{\otimes m} \arrow[d, ustred, dashed, "\tau"] \\
    & V^{\otimes (n + m)}
  \end{tikzcd}
\end{center}
Here, the blue arrows represent the construction of the tensor products $V^{\otimes n} \otimes V^{\otimes m}$. The red arrows represent the universal property of the tensor product $V^{\otimes (n + m)}$. The dashed red arrow $\tau$ is the unique bilinear map that makes the diagram commute. Then we have shown the well-definedness of the multiplication map $\otimes \colon \T(V) \times \T(V) \to \T(V)$.

The tensor algebra $(\T(V), \otimes)$ is a graded unital associative algebra over $F$.
\begin{itemize}
  \item It is graded, as it is \emph{degree additive}, i.e., $V^{\otimes n} \times V^{\otimes m} \to V^{\otimes (n + m)}$ for all $n, m \geq 0$;
  \item It is unital, as the multiplicative identity is 1 in $F$, since $V^{\otimes 0} = F$;
  \item It is associative, as $(u \otimes v) \otimes w = u \otimes (v \otimes w)$ for all $u, v, w \in V$ and the associativity can be extended to all homogeneous elements by bi-additivity;
  \item It is non-commutative, as $u \otimes v \neq v \otimes u$ for some $u, v \in V$.
\end{itemize}

Similarly, we can characterise tensor algebras via a universal property.
\begin{proposition}[Universal Property of Tensor Algebra]\label{prop:universal_property_tensor_algebra}
  Let $V$ be a finite-dimensional $F$-linear space. For any graded unital associative $F$-algebra $(A^{\smallbullet}, \cdot)$ and any graded linear map with graded degree zero $\phi \colon V \to A^{\smallbullet}$, there exists a unique graded algebra homomorphism with graded degree zero $\widetilde{\phi} \colon \T(V) \to A^{\smallbullet}$ such that the following diagram commutes:
  \begin{center}
    \begin{tikzcd}
      V \arrow[r, hook, "\iota"] \arrow[rd, "\phi"] & \T(V) \arrow[d, dashed, "\widetilde{\phi}"] \\
      & A^{\smallbullet}
    \end{tikzcd}
  \end{center}
  More specifically, here $\iota$ is a map that includes $V$ into the degree $1$ part of $\T(V)$, i.e., $\iota \colon V \to V^{\otimes 1} = V$, and $\phi$ is a map from $V$ to the degree $1$ part of $A^{\smallbullet}$, i.e., $\phi \colon V \to A_1$, as we can consider $V$ as a graded linear space concentrated in degree $1$.
\end{proposition}
\begin{proof}
  If such a map $\widetilde{\phi}$ exists, then for any $v \in V$, we must have
  \[
    \widetilde{\phi}(\iota(v)) = \widetilde{\phi}(0, v, 0, 0, \ldots) = (0, \phi(v), 0, 0, \ldots) = \phi(v).
  \]
  Since elements of the form $v \in V$ generate $\T(V)$ as an algebra, this determines $\widetilde{\phi}$ uniquely. To show existence, we define $\widetilde{\phi}$ on the basis elements by $\widetilde{\phi}(v_1 \otimes v_2 \otimes \cdots \otimes v_k) := \phi(v_1) \cdot \phi(v_2) \cdots \phi(v_k)$ for all $v_1, v_2, \ldots, v_k \in V$, and extend linearly to all of $\T(V)$. It is straightforward to verify that $\widetilde{\phi}$ is a graded algebra homomorphism with graded degree zero and makes the diagram commute.
\end{proof}

This proposition shows that any graded linear map with graded degree zero from $V$ to a graded unital associative $F$-algebra $A^{\smallbullet}$ can be uniquely factored through the tensor algebra $\T(V)$, i.e., $\Vect_F(V, A^{\smallbullet}) \simeq \Alg(\T(V), A^{\smallbullet})$. Moreover, we have the natural isomorphism between $\Alg(\T(V), -)$ and $\Vect_F(V, -)$:
\[
  \Vect_F(V, |-|) \simeq \Alg(\T(V), -).
\]
This shows that the tensor algebra functor $\T$ is left adjoint to the forgetful functor from the category of graded unital associative algebras over $F$ to the category of linear spaces over $F$:
\begin{center}
  \begin{tikzcd}[labels=auto]
    \Vect_F \arrow[r, "\T", yshift=0.5ex] & \Z_{\geq 0} - \Alg_F \arrow[l, "|-|", yshift=-0.5ex]
  \end{tikzcd}
\end{center}
Such an adjunction is called a \emph{free-forgetful adjunction}. We can also see the tensor algebra as the \emph{free graded unital associative algebra} generated by the linear space $V$:
\begin{center}
  \begin{tikzcd}[row sep=normal]
    \Vect_F \arrow[r, "\T"] & \Z_{\geq 0} - \Alg_F \\[-1.8em]
    V \arrow[dd, "f"] & \T(V) \arrow[dd, "\T(f)"] \\
    \arrow[r, mapsto, shorten <= 2ex, shorten >= 2ex] & \phantom{*} \\
    W & \T(W)
  \end{tikzcd}
\end{center}
Here, $f \colon V \to W$ is a linear map between linear spaces, and $\T(f) \colon \T(V) \to \T(W)$ is the induced graded algebra homomorphism with graded degree zero between tensor algebras.

\subsection{Quotient Algebras}
We can construct new algebras from existing algebras using quotienting. We will discuss three types of quotient algebras: the symmetric algebra, the exterior algebra, and the universal enveloping algebra. Before that, we first define ideals in algebras.

\begin{definition}[Ideal]\label{def:ideal_in_algebra}
  An \emph{ideal} $I$ in an algebra $(A, \mu)$ over $F$ is a \hyperref[def:linear_subspace]{linear subspace} of $A$ that respects the \hyperref[def:algebraic_structure]{algebraic structure}; that is, for all $a \in A$ and $x \in I$, the following properties hold:
  \[
    \mu(a, x) \in I, \quad \mu(x, a) \in I.
  \]
\end{definition}

This ideal definition is the restriction of the general ideal definition in rings to algebras over fields. We can consider the following example for general ideals in rings.
\begin{example}
  In the ring of integers $\Z$, the set of all $n$-multiples $n\Z$ forms an ideal for any integer $n$. This ideal respects the ringic structure, as for any integer $a$ and any $x \in n\Z$, we have $a \cdot x \in n\Z$ and $x \cdot a \in n\Z$.
\end{example}

\subsubsection{Symmetric Algebras}
As the name suggests, symmetric algebras are algebras that are `commutative' in some sense.

\begin{definition}[Symmetric Algebra]\label{def:symmetric_algebra}
  The \emph{symmetric algebra} of a finite-dimensional $F$-linear space $V$, denoted by $\Sym(V)$, is the quotient algebra of the \hyperref[def:tensor_algebra]{tensor algebra} $\T(V)$ by the \emph{symmetrising \hyperref[def:ideal_in_algebra]{ideal}} $I_{\Sym}$ generated by all elements of the form $v \otimes w - w \otimes v$ for all $v, w \in V$:
  \[
    \Sym(V) = \quotient{\T(V)}{I_{\Sym}} = \quotient{\T(V)}{\langle v \otimes w - w \otimes v \mid v, w \in V \rangle}.
  \]
\end{definition}
The symmetrising ideal is the ideal completion of the relation $v \otimes w = w \otimes v$ for all $v, w \in V$, which enforces the commutativity condition in the symmetric algebra. Thus, the symmetric algebra is a graded unital commutative associative algebra over $F$. We can also characterise symmetric algebras via a universal property.

\begin{proposition}[Universal Property of Symmetric Algebra]\label{prop:universal_property_symmetric_algebra}
  Let $V$ be a finite-dimensional $F$-linear space. For any graded unital commutative associative $F$-algebra $(A^{\smallbullet}, \cdot)$ and any graded linear map with graded degree zero $\phi \colon V \to A^{\smallbullet}$, there exists a unique graded algebra homomorphism with graded degree zero $\widetilde{\phi} \colon \Sym(V) \to A^{\smallbullet}$ such that the following diagram commutes:
  \begin{center}
    \begin{tikzcd}[column sep=normal]
      V \arrow[r, hook, "\iota"] \arrow[rrd, "\phi"] & \T(V) \arrow[r, two heads, "\pi"] & \Sym(V) \arrow[d, dashed, "\widetilde{\phi}"] \\
      & & A^{\smallbullet}
    \end{tikzcd}
  \end{center}
  More specifically, here $\iota$ is a map that includes $V$ into the degree $1$ part of $\Sym(V)$, i.e., $\iota \colon V \to V^{\otimes 1} = V$, and $\phi$ is a map from $V$ to the degree $1$ part of $A^{\smallbullet}$, i.e., $\phi \colon V \to A_1$, as we can consider $V$ as a graded linear space concentrated in degree $1$.
\end{proposition}

This proposition shows that any graded linear map with graded degree zero from $V$ to a graded unital commutative associative $F$-algebra $A^{\smallbullet}$ can be uniquely factored through the symmetric algebra $\Sym(V)$, i.e., $\Vect_F(V, A^{\smallbullet}) \simeq \CAlg(\Sym(V), A^{\smallbullet})$. Moreover, we have the natural isomorphism between $\CAlg(\Sym(V), -)$ and $\Vect_F(V, -)$:
\[
  \Vect_F(V, |-|) \simeq \CAlg(\Sym(V), -).
\]
This shows that the symmetric algebra functor $\Sym$ is left adjoint to the forgetful functor from the category of graded unital commutative associative algebras over $F$ to the category of linear spaces over $F$:
\begin{center}
  \begin{tikzcd}[labels=auto]
    \Vect_F \arrow[r, "\Sym", yshift=0.5ex] & \Z_{\geq 0} - \CAlg_F \arrow[l, "|-|", yshift=-0.5ex]
  \end{tikzcd}
\end{center}
We can also see the symmetric algebra as the \emph{free graded unital commutative associative algebra} generated by the linear space $V$:
\begin{center}
  \begin{tikzcd}[row sep=normal]
    \Vect_F \arrow[r, "\Sym"] & \Z_{\geq 0} - \CAlg_F \\[-1.8em]
    V \arrow[dd, "f"] & \Sym(V) \arrow[dd, "\Sym(f)"] \\
    \arrow[r, mapsto, shorten <= 2ex, shorten >= 2ex] & \phantom{*} \\
    W & \Sym(W)
  \end{tikzcd}
\end{center}
Here, $f \colon V \to W$ is a linear map between linear spaces, and $\Sym(f) \colon \Sym(V) \to \Sym(W)$ is the induced graded algebra homomorphism with graded degree zero between symmetric algebras.

\begin{definition}[Symmetric Power]\label{def:symmetric_power}
  The \emph{$k$-th symmetric power} of a finite-dimensional $F$-linear space $V$, denoted by $\Sy^k(V)$, is defined as the degree $k$ component of the \hyperref[def:symmetric_algebra]{symmetric algebra} $\Sym(V)$:
  \[
    \Sym(V) = \bigoplus_{k = 0}^{\infty} \Sy^k(V).
  \]
\end{definition}
The dimension of the $k$-th symmetric power is given by the following formula:
\[ \dim(\Sy^k(V)) = \binom{\dim(V) + k - 1}{k}. \]
Similarly, we can characterise symmetric powers via a universal property.

\begin{proposition}[Universal Property of Symmetric Power]\label{prop:universal_property_symmetric_power}
  Let $V$ be a finite-dimensional $F$-linear space. For any finite-dimensional $F$-linear space $W$ and any symmetric bilinear map $\phi \colon V^k \to W$ ($k$ times), there exists a unique linear map $\widetilde{\phi} \colon \Sy^k(V) \to W$ such that the following diagram commutes:
  \begin{center}
    \begin{tikzcd}[column sep=normal]
      V^k \arrow[r, hook, "\iota"] \arrow[rrd, "\phi"] & V^{\otimes k} \arrow[r, two heads, "\pi"] & \Sy^k(V) \arrow[d, dashed, "\widetilde{\phi}"] \\
      & & W
    \end{tikzcd}
  \end{center}
\end{proposition}

\subsubsection{Exterior Algebras}
Exterior algebras are algebras that are `skew-symmetric' in some sense.

\begin{definition}[Exterior Algebra]\label{def:exterior_algebra}
  The \emph{exterior algebra} of a finite-dimensional $F$-linear space $V$, denoted by $\Ext(V)$, is the quotient algebra of the \hyperref[def:tensor_algebra]{tensor algebra} $\T(V)$ by the \emph{antisymmetrising \hyperref[def:ideal_in_algebra]{ideal}} $I_{\Ext}$ generated by all elements of the form $v \otimes w + w \otimes v$ for all $v, w \in V$:
  \[
    \Ext(V) = \quotient{\T(V)}{I_{\Ext}} = \quotient{\T(V)}{\langle v \otimes w + w \otimes v \mid v, w \in V \rangle}.
  \]
\end{definition}
\begin{remark}
  Here, we assume that the characteristic of the field $F$ is not 2, so that the relation $v \otimes w + w \otimes v = 0$ is equivalent to $v \otimes w = - (w \otimes v)$ for all $v, w \in V$. If the characteristic of $F$ is 2, then the antisymmetrising ideal would be generated by all elements of the form $v \otimes v$ for all $v \in V$.
\end{remark}
The antisymmetrising ideal is the ideal completion of the relation $v \otimes w = - (w \otimes v)$ for all $v, w \in V$, which enforces the skew-symmetry condition in the exterior algebra. Thus, the exterior algebra is a graded unital associative algebra over $F$. We can also characterise exterior algebras via a universal property.

\begin{proposition}[Universal Property of Exterior Algebra]\label{prop:universal_property_exterior_algebra}
  Let $V$ be a finite-dimensional $F$-linear space. For any graded unital associative $F$-algebra $(A^{\smallbullet}, \cdot)$ and any graded linear map with graded degree zero $\phi \colon V \to A^{\smallbullet}$ satisfying $\phi(v) \cdot \phi(v) = 0$ for all $v \in V$, there exists a unique graded algebra homomorphism with graded degree zero $\widetilde{\phi} \colon \Ext(V) \to A^{\smallbullet}$ such that the following diagram commutes:
  \begin{center}
    \begin{tikzcd}[column sep=normal]
      V \arrow[r, hook, "\iota"] \arrow[rrd, "\phi"] & \T(V) \arrow[r, two heads, "\pi"] & \Ext(V) \arrow[d, dashed, "\widetilde{\phi}"] \\
      & & A^{\smallbullet}
    \end{tikzcd}
  \end{center}
  More specifically, here $\iota$ is a map that includes $V$ into the degree $1$ part of $\Ext(V)$, i.e., $\iota \colon V \to V^{\otimes 1} = V$, and $\phi$ is a map from $V$ to the degree $1$ part of $A^{\smallbullet}$, i.e., $\phi \colon V \to A_1$, as we can consider $V$ as a graded linear space concentrated in degree $1$.
\end{proposition}

This proposition shows that any graded linear map with graded degree zero from $V$ to a graded unital associative $F$-algebra $A^{\smallbullet}$ that satisfies the condition $\phi(v) \cdot \phi(v) = 0$ for all $v \in V$ can be uniquely factored through the exterior algebra $\Ext(V)$, i.e., $\Vect_F(V, A^{\smallbullet}) \simeq \Alg(\Ext(V), A^{\smallbullet})$. Moreover, we have the natural isomorphism between $\SAlg(\Ext(V), -)$ and $\Vect_F(V, -)$:
\[
  \Vect_F(V, |-|) \simeq \SAlg(\Ext(V), -).
\]
This shows that the exterior algebra functor $\Ext$ is left adjoint to the forgetful functor from the category of graded unital associative algebras over $F$ to the category of linear spaces over $F$:
\begin{center}
  \begin{tikzcd}[labels=auto]
    \Vect_F \arrow[r, "\Ext", yshift=0.5ex] & \Z_{\geq 0} - \SAlg_F \arrow[l, "|-|", yshift=-0.5ex]
  \end{tikzcd}
\end{center}
We can also see the exterior algebra as the \emph{free graded unital skew-symmetric associative algebra} generated by the linear space $V$:
\begin{center}
  \begin{tikzcd}[row sep=normal]
    \Vect_F \arrow[r, "\Ext"] & \Z_{\geq 0} - \SAlg_F \\[-1.8em]
    V \arrow[dd, "f"] & \Ext(V) \arrow[dd, "\Ext(f)"] \\
    \arrow[r, mapsto, shorten <= 2ex, shorten >= 2ex] & \phantom{*} \\
    W & \Ext(W)
  \end{tikzcd}
\end{center}
Here, $f \colon V \to W$ is a linear map between linear spaces, and $\Ext(f) \colon \Ext(V) \to \Ext(W)$ is the induced graded algebra homomorphism with graded degree zero between exterior algebras.

\begin{definition}[Exterior Power]\label{def:exterior_power}
  The \emph{$k$-th exterior power} of a finite-dimensional $F$-linear space $V$, denoted by $\Lambda^k(V)$, is defined as the degree $k$ component of the \hyperref[def:exterior_algebra]{exterior algebra} $\Ext(V)$:
  \[
    \Ext(V) = \bigoplus_{k = 0}^{\infty} \Lambda^k(V).
  \]
\end{definition}
The dimension of the $k$-th exterior power is given by the following formula:
\[ \dim(\Lambda^k(V)) = \binom{\dim(V)}{k} \]
Similarly, we can characterise exterior powers via a universal property.

\begin{proposition}[Universal Property of Exterior Power]\label{prop:universal_property_exterior_power}
  Let $V$ be a finite-dimensional $F$-linear space. For any finite-dimensional $F$-linear space $W$ and any skew-symmetric bilinear map $\phi \colon V^k \to W$ ($k$ times), there exists a unique linear map $\widetilde{\phi} \colon \Lambda^k(V) \to W$ such that the following diagram commutes:
  \begin{center}
    \begin{tikzcd}[column sep=normal]
      V^k \arrow[r, hook, "\iota"] \arrow[rrd, "\phi"] & V^{\otimes k} \arrow[r, two heads, "\pi"] & \Lambda^k(V) \arrow[d, dashed, "\widetilde{\phi}"] \\
      & & W
    \end{tikzcd}
  \end{center}
\end{proposition}

\subsubsection{Universal Enveloping Algebras}
Universal enveloping algebras are algebras that `envelop' Lie algebras in some sense.

\begin{definition}[Universal Enveloping Algebra]\label{def:universal_enveloping_algebra}
  The \emph{universal enveloping algebra} of a finite-dimensional Lie algebra $\mathfrak{g}$ over $F$, denoted by $\mathcal{U}(\mathfrak{g})$, is the quotient algebra of the \hyperref[def:tensor_algebra]{tensor algebra} $\T(\mathfrak{g})$ by the \emph{Lie ideal} $I_{\mathcal{U}}$ generated by all elements of the form $x \otimes y - y \otimes x - [x, y]$ for all $x, y \in \mathfrak{g}$:
  \[
    \mathcal{U}(\mathfrak{g}) = \quotient{\T(\mathfrak{g})}{I_{\mathcal{U}}} = \quotient{\T(\mathfrak{g})}{\langle x \otimes y - y \otimes x - [x, y] \mid x, y \in \mathfrak{g} \rangle}.
  \]
\end{definition}
The Lie ideal is the ideal completion of the relation $x \otimes y - y \otimes x = [x, y]$ for all $x, y \in \mathfrak{g}$, which enforces the Lie bracket condition in the universal enveloping algebra. Thus, the universal enveloping algebra is a unital associative algebra over $F$. However, it is not graded, as the ideal is not homogeneous with respect to the grading of the tensor algebra: $x \otimes y - y \otimes x$ is in degree 2, while $[x, y]$ is in degree 1.

\subsection{Hilbert-Poincaré Series}
We use Hilbert-Poincaré series to encode the dimension information of graded linear spaces.

\begin{definition}[Hilbert-Poincaré Series]\label{def:hilbert_poincare_series}
  Let $V_{\smallbullet} = \bigoplus_{n \in \Z_{\geq 0}} V_n$ be a $\Z_{\geq 0}$-\hyperref[def:graded_linear_space]{graded linear space} over $F$ with each degree component $V_n$ being finite-dimensional. The \emph{Hilbert-Poincaré series} of $V_{\smallbullet}$ is the formal power series defined as follows:
  \[
    P_{V_{\smallbullet}}(t) = \sum_{n = 0}^{\infty} (\dim(V_n)) t^n.
  \]
\end{definition}

\begin{example}
  The Hilbert-Poincaré series of the tensor algebra $\T(V)$ of a finite-dimensional $F$-linear space $V$ is given by the following formula:
  \[
    P_{\T(V)}(t) = \sum_{n = 0}^{\infty} (\dim(V^{\otimes n})) = \sum_{n = 0}^{\infty} {(\dim(V))}^n t^n = \frac{1}{1 - (\dim(V)) t}.
  \]
\end{example}

\begin{example}
  The Hilbert-Poincaré series of the symmetric algebra $\Sym(V)$ of a finite-dimensional $F$-linear space $V$ is given by the following formula:
  \[
    P_{\Sym(V)}(t) = \sum_{n = 0}^{\infty} (\dim(\Sym^n(V))) = \sum_{n = 0}^{\infty} \binom{\dim(V) + n - 1}{n} t^n = \frac{1}{{(1 - t)}^{\dim(V)}}.
  \]
\end{example}

\begin{example}
  The Hilbert-Poincaré series of the exterior algebra $\Ext(V)$ of a finite-dimensional $F$-linear space $V$ is given by the following formula:
  \[
    P_{\Ext(V)}(t) = \sum_{n = 0}^{\infty} (\dim(\Lambda^n(V))) = \sum_{n = 0}^{\infty} \binom{\dim(V)}{n} t^n = {(1 + t)}^{\dim(V)}.
  \]
\end{example}

As the Hilbert-Poincaré series of the exterior algebra is a polynomial of degree $\dim(V)$, we have $\Lambda^k(V) = 0$ for all $k > \dim(V)$. Moreover, if $\dim(V) = n$, then $\Lambda^n(V)$ is a 1-dimensional linear space called the \emph{top exterior power} of $V$. The reason is that any $n + 1$ vectors in an $n$-dimensional linear space are linearly dependent, so their exterior product is zero. Also, $\dim(\Lambda^k(V)) = \dim(\Lambda^{n - k}(V))$ for all $0 \leq k \leq n$, as $\binom{n}{k} = \binom{n}{n - k}$.

\begin{definition}[Line]\label{def:line}
  A \emph{line} over $F$ is a 1-dimensional linear space over $F$.
\end{definition}

\clearpage{}

\section{Exercises}

\begin{problem}
Show that there is a functor $\otimes$ from the product category $\Vect{\F}^{\fd} \times \Vect{\F}^{\fd}$ to the category $\Vect{\F}^{\fd}$ that sends an object $(V_1, V_2)$ to $V_1 \otimes V_2$ and a morphism $(f, g)$ to $f \otimes g$. Show that $f \otimes g$ is bilinear, meaning it is linear in both $f$ and $g$. Finally, show that the functors $- \otimes V$, $\Hom(V, -)$ and $\Hom(-, V)$ preserve exactness: If $A \to B \to C$ is exact, then the sequence $A \otimes V \to B \otimes V \to C \otimes V$, $\Hom(V, A) \to \Hom(V, B) \to \Hom(V, C)$ and $\Hom(A, V) \leftarrow \Hom(B, V) \leftarrow \Hom(C, V)$ are exact.
\end{problem}

\begin{problem}
\begin{enumerate}
  \item Show that functors ${(-)}^{**}$, $- \otimes \F$, $\F \otimes -$, $\Hom(\F, -)$ and the identity functor 1 are all naturally equivalent endofunctors on the category $\Vect{\F}^{\fd}$. A simpler way to record these facts is to write
        \[
          V^{**} \equiv V \otimes \F \equiv \F \otimes V \equiv \Hom(\F, V) \equiv V
        \]

  \item Show that
        \[
          \Hom(V_1, V_2 \otimes V_3) \equiv \Hom(V_1, V_2) \otimes V_3.
        \]
        Consequently, $\Hom(V_1, V_2) \equiv V_1^* \otimes V_2$ and ${(V_1 \otimes V_2)}^* \equiv V_1^* \otimes V_2^*$.

  \item $\dim V_1 \otimes V_2 = \dim V_1 \cdot \dim V_2$. Moreover, if $e_i$ is a minimal spanning set of $V_1$ and $f_j$ is a minimal spanning set of $V_2$, then $e_i \otimes f_j$ is a minimal spanning set of $V_1 \otimes V_2$.

  \item Show that
        \[
          V_1 \otimes V_2 \equiv V_2 \otimes V_1, \qquad \End V \equiv {(\End V)}^*
        \]

  \item Under the natural identification $\End V \equiv {(\End V)}^*$, $\id_V$ is identified with a linear map $\tr: \End V \to \F$. Show that $\tr$ is cyclic, i.e. $\tr(TS)$ = $\tr(ST)$, and $\tr\id_V = \dim V$. This map is called the trace map.
\end{enumerate}
\end{problem}

\begin{problem}
Let the category $\Vect{\F}^{\fd}$ be denoted by $\mathcal{V}$. Denote by $\mathcal{V}^{\mathsf{op}}$ be the opposite category of $\mathcal{V}$. Show that
\begin{enumerate}
  \item $V_1 \otimes (V_2 \oplus V_3) \equiv (V_1 \otimes V_2) \oplus (V_1 \otimes V_3)$. This is a natural equivalene of two functors from $\mathcal{V} \times \mathcal{V} \times \mathcal{V}$ to $\mathcal{V}$;
  \item $\Hom(V_1, V_2 \oplus V_3) \equiv \Hom(V_1, V_2) \oplus \Hom(V_1, V_3)$. This is a natural equivalene of two functors from $\mathcal{V}^{\mathsf{op}} \times \mathcal{V} \times \mathcal{V}$ to $\mathcal{V}$;
  \item $\Hom(V_1 \oplus V_2, V_3) \equiv \Hom(V_1, V_3) \times \Hom(V_2, V_3)$. This is a natural equivalene of two functors from $\mathcal{V}^{\mathsf{op}} \times \mathcal{V}^{\mathsf{op}} \times \mathcal{V}$ to $\mathcal{V}$.
\end{enumerate}
\end{problem}