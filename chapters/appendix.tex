\chapter*{Appendix: Fudan University Problems}\label{appendix:fudan_problems}

Students from Fudan University asked two hard problems but were completely cooked by Professor Guowu Meng.

\textbf{The story behind the two problems.}

``Well, [in] linear algebra basically, no problem is difficult. All problems are trivial.

``People don't believe me, because many years ago, more than 20 years ago, there were two exchange students from Fudan University, and when they came here, they carry solution manual with some sets of hard linear algebra problems. I told them `nothing is difficult'.

``They don't believe me, so they dig out one hard problem from that solution book. Well, I told them I haven't seen this problem before, because when I was educated as a physicist engineer, I don't work on hard problems. I just deal with textbook. I don't read anything extra. I don't know but doesn't matter. Let me just write everything on board, and then pretty soon I figured out the answer.

``Ok may be they say that I am lucky. Then the next day they came back with another problem. So again, I said I don't know how to do it but anyway [it] doesn't matter. I put everything on board, then I draw some obvious facts in my mind about linear algebra.

``I say no problems are difficult in linear algebra under the assumption that you know linear algebra inside-out, you know every facts about it. Usually you will say I have seen this type of problems before, and then step 1, step 2 step 3, but this is a very wrong way to do it. This is the way that AI does it, but we are human, we are smarter than machine.

``When I do it, there are some keywords and each keywords remind me of some facts related to it, and keep doing this. Then I see a path from here to there.''

\begin{flushright}
	--- Guowu Meng on the lecture of September 19, 2025.
\end{flushright}

\noindent\textbf{Problem 1.}
Suppose we have three matrices $A$, $B$ and $C$. Then prove that
\[
	\rank(B) + \rank(ABC) \geq \rank(AB) + \rank(BC)
\]
\textbf{Solution.} We consider the following diagram:
\begin{center}
	\begin{tikzcd}[row sep=huge, column sep=large]
		0 \arrow[r] &
		| [alias=D] | \col(BC) \arrow[r, hook, "C" description, red] \arrow[d, two heads, "A" description, blue] \arrow[rdr, start anchor=south, rounded corners, to path={
					-- (B.south)
					-- (\tikztotarget.north west)
				}, orange, thick, -Stealth] \arrow[drr, start anchor=south, rounded corners, to path={
					-- (D.south west)
					|- (\tikztotarget.south west)
				}, violet, thick, -Stealth] &
		| [alias=B] | \col(B) \arrow[r, two heads, "\pi_1" description, blue] \arrow[d, two heads, "A" description, blue] \arrow[dr, two heads, teal] &
		\quotient{\col(B)}{\col(BC)} \arrow[r] \arrow[d, two heads, "\exists ! \phi" description, dashed, ustblue] &
		0 \\

		0 \arrow[r] &
		| [alias=C] | \col(ABC) \arrow[r, hook, "C" description, red] &
		| [alias=A] | \col(AB) \arrow[r, two heads, "\pi_2" description, blue] &
		\quotient{\col(AB)}{\col(ABC)} \arrow[r] &
		0
	\end{tikzcd}
\end{center}

We denote the injective map with red color and the surjective map with blue color. Notice that there is a surjective map from $\col(B)$ to $\quotient{\col(AB)}{\col(ABC)}$ due to the surjectivity of $A$ and $\pi_2$. Then we denote this surjective map with teal color.

Then we have to consider whether the map from $\col(BC)$ to $\quotient{\col(AB)}{\col(ABC)}$ is zero. If the map is zero, then we can construct a unique surjective map $\phi$ from $\quotient{\col(B)}{\col(BC)}$ to $\quotient{\col(AB)}{\col(ABC)}$ due to the universal property of quotient space.

Note that the map from $\col(BC)$ to $\quotient{\col(AB)}{\col(ABC)}$ is a zero map. As both upper and lower sequences are exact, we have the exactness at $\col(AB)$, i.e., $\im{C} = \ker{\pi_2}$. Thus the composite map $\pi_2 \circ C$ is a zero map. This shows that the map from $\col(BC)$ to $\quotient{\col(AB)}{\col(ABC)}$ is a zero map.

Then we can construct a unique surjective map $\phi$ from $\quotient{\col(B)}{\col(BC)}$ to $\quotient{\col(AB)}{\col(ABC)}$ due to the universal property of quotient space.

Finally, we consider the dimensions of the spaces. Note that $\phi$ is surjective, thus we have
\begin{align*}
	\dim(\quotient{\col(B)}{\col(BC)}) & \geq \dim(\quotient{\col(AB)}{\col(ABC)}) \\
	\dim(\col(B)) - \dim(\col(BC))     & \geq \dim(\col(AB)) - \dim(\col(ABC))     \\
	\dim(\col(B)) + \dim(\col(ABC))    & \geq \dim(\col(AB)) + \dim(\col(BC))      \\
	\rank(B) + \rank(ABC)              & \geq \rank(AB) + \rank(BC) \qedhere
\end{align*}

\noindent\textbf{Problem 2.}
If $A$ is a $n \times n$ matrix then prove that
\[
	\rank(A^n) = \rank(A^{n+1})
\]
\textbf{Solution.} We consider the following diagram:
\begin{center}
	\begin{tikzcd}[column sep=normal]
		I_n \arrow[r, "A"] & \im(A) \arrow[r, "A"] & \im(A^2) \arrow[r, "A"] & \cdots \arrow[r, "A"] & \im(A^n) \arrow[r, "A"] & \cdots
	\end{tikzcd}
\end{center}
As $I_n \supseteq \im(A) \supseteq \im(A^2) \supseteq \cdots$, we know that
\[
	n = \dim(I_n) \geq \rank(A) \geq \rank(A^2) \geq \cdots
\]

As the space is finite-dimensional, the sequence will eventually become constant. That means there exists a $k$ such that for all $j \geq k$, we have $\rank(A^j) = \rank(A^{j + 1})$.

There are two possibilities: either $k \leq n$ or $k > n$. If $k \leq n$, the equality works properly, as for every $j \geq k$, including $j = n$, such that $\rank(A^j) = \rank(A^{j + 1})$ implies $\rank(A^n) = \rank(A^{n + 1})$.

For $k > n$, consider the strict inequality, we know that each time the dimension must drop at least 1. Without the loss of generality, we may consider the sequence of dimension as $n, n - 1, n - 2, \cdots, 1, 0$. This involves $n$ times. So it is impossible to have $k > n$.